% Options for packages loaded elsewhere
\PassOptionsToPackage{unicode}{hyperref}
\PassOptionsToPackage{hyphens}{url}
%
\documentclass[
]{book}
\usepackage{amsmath,amssymb}
\usepackage{iftex}
\ifPDFTeX
  \usepackage[T1]{fontenc}
  \usepackage[utf8]{inputenc}
  \usepackage{textcomp} % provide euro and other symbols
\else % if luatex or xetex
  \usepackage{unicode-math} % this also loads fontspec
  \defaultfontfeatures{Scale=MatchLowercase}
  \defaultfontfeatures[\rmfamily]{Ligatures=TeX,Scale=1}
\fi
\usepackage{lmodern}
\ifPDFTeX\else
  % xetex/luatex font selection
\fi
% Use upquote if available, for straight quotes in verbatim environments
\IfFileExists{upquote.sty}{\usepackage{upquote}}{}
\IfFileExists{microtype.sty}{% use microtype if available
  \usepackage[]{microtype}
  \UseMicrotypeSet[protrusion]{basicmath} % disable protrusion for tt fonts
}{}
\makeatletter
\@ifundefined{KOMAClassName}{% if non-KOMA class
  \IfFileExists{parskip.sty}{%
    \usepackage{parskip}
  }{% else
    \setlength{\parindent}{0pt}
    \setlength{\parskip}{6pt plus 2pt minus 1pt}}
}{% if KOMA class
  \KOMAoptions{parskip=half}}
\makeatother
\usepackage{xcolor}
\usepackage{longtable,booktabs,array}
\usepackage{calc} % for calculating minipage widths
% Correct order of tables after \paragraph or \subparagraph
\usepackage{etoolbox}
\makeatletter
\patchcmd\longtable{\par}{\if@noskipsec\mbox{}\fi\par}{}{}
\makeatother
% Allow footnotes in longtable head/foot
\IfFileExists{footnotehyper.sty}{\usepackage{footnotehyper}}{\usepackage{footnote}}
\makesavenoteenv{longtable}
\usepackage{graphicx}
\makeatletter
\def\maxwidth{\ifdim\Gin@nat@width>\linewidth\linewidth\else\Gin@nat@width\fi}
\def\maxheight{\ifdim\Gin@nat@height>\textheight\textheight\else\Gin@nat@height\fi}
\makeatother
% Scale images if necessary, so that they will not overflow the page
% margins by default, and it is still possible to overwrite the defaults
% using explicit options in \includegraphics[width, height, ...]{}
\setkeys{Gin}{width=\maxwidth,height=\maxheight,keepaspectratio}
% Set default figure placement to htbp
\makeatletter
\def\fps@figure{htbp}
\makeatother
\setlength{\emergencystretch}{3em} % prevent overfull lines
\providecommand{\tightlist}{%
  \setlength{\itemsep}{0pt}\setlength{\parskip}{0pt}}
\setcounter{secnumdepth}{5}
\newlength{\cslhangindent}
\setlength{\cslhangindent}{1.5em}
\newlength{\csllabelwidth}
\setlength{\csllabelwidth}{3em}
\newlength{\cslentryspacingunit} % times entry-spacing
\setlength{\cslentryspacingunit}{\parskip}
\newenvironment{CSLReferences}[2] % #1 hanging-ident, #2 entry spacing
 {% don't indent paragraphs
  \setlength{\parindent}{0pt}
  % turn on hanging indent if param 1 is 1
  \ifodd #1
  \let\oldpar\par
  \def\par{\hangindent=\cslhangindent\oldpar}
  \fi
  % set entry spacing
  \setlength{\parskip}{#2\cslentryspacingunit}
 }%
 {}
\usepackage{calc}
\newcommand{\CSLBlock}[1]{#1\hfill\break}
\newcommand{\CSLLeftMargin}[1]{\parbox[t]{\csllabelwidth}{#1}}
\newcommand{\CSLRightInline}[1]{\parbox[t]{\linewidth - \csllabelwidth}{#1}\break}
\newcommand{\CSLIndent}[1]{\hspace{\cslhangindent}#1}
\usepackage{booktabs}
\usepackage{amsthm}
\makeatletter
\def\thm@space@setup{%
  \thm@preskip=8pt plus 2pt minus 4pt
  \thm@postskip=\thm@preskip
}
\makeatother
\ifLuaTeX
  \usepackage{selnolig}  % disable illegal ligatures
\fi
\IfFileExists{bookmark.sty}{\usepackage{bookmark}}{\usepackage{hyperref}}
\IfFileExists{xurl.sty}{\usepackage{xurl}}{} % add URL line breaks if available
\urlstyle{same}
\hypersetup{
  pdftitle={River Herring Habitat in the Northeast United States},
  pdfauthor={Vanessa Quintana, Justin Stevens, Kyle McKay, Andrew Jacobs, \ldots.. (other interested collaborators)},
  hidelinks,
  pdfcreator={LaTeX via pandoc}}

\title{River Herring Habitat in the Northeast United States}
\author{Vanessa Quintana, Justin Stevens, Kyle McKay, Andrew Jacobs, \ldots.. (other interested collaborators)}
\date{2024-02-20}

\begin{document}
\maketitle

{
\setcounter{tocdepth}{1}
\tableofcontents
}
\hypertarget{preface}{%
\chapter*{Preface}\label{preface}}
\addcontentsline{toc}{chapter}{Preface}

\hypertarget{abstract}{%
\section{Abstract}\label{abstract}}

\hypertarget{acknowledgments}{%
\section{Acknowledgments}\label{acknowledgments}}

\begin{itemize}
\tightlist
\item
  Wampanoag Tribe
\item
  New England USACE
\item
  NOAA SeaGrant
\end{itemize}

\hypertarget{intro}{%
\chapter{Introduction}\label{intro}}

\begin{itemize}
\item
  A brief history of river herring management efforts.
\item
  Challenges and limitations faced in historical management practices.
\item
  The ecological significance of river herring in aquatic ecosystems.
\end{itemize}

\hypertarget{history-of-river-herring}{%
\section{History of River Herring}\label{history-of-river-herring}}

\emph{map of river herring territory}

River herring is a term used to generalize diadromous cluepid species in the Northeast United States. Specifically, Alewives and Blueback herring are known collectively as River herring and are a foundational resource that has played a pivotal role in the development of many communities past and present. Before the founding of the United States, indigenous communities utilized river herring for food an fertilizer (Wampanoag). Early settlers learned how to fish river herring from these indigenous communities. During early settlement, waterways were significantly altered and commercial fishing practices grew, which eventually divulged into the exploitation of the herring stock and species as a whole. Throughout the industrial period, waterways were full of migration impedements while pollution decreased water quality. Commercial fishing was an all time high and bycatch during offshore fishing has been noted as a significant factor in the decrease of river herring ().

\hypertarget{ecological-significance-of-herring}{%
\section{Ecological Significance of herring}\label{ecological-significance-of-herring}}

River herring act as a significant flux of nutrients and energy into the terrestrial aquatic systems. This influx of energy is pivitol to these aquatic systems and provides a food resource and attract game fish to the system.

\hypertarget{ecological-modeling-in-river-herring-management}{%
\section{Ecological Modeling in River Herring Management}\label{ecological-modeling-in-river-herring-management}}

Existing river herring habitat suitability models, originally developed by Brown et al. (2000) and Pardue (1983), with reliance on similar sources such as Bigelow and Schroeder (1953) and the updated version Collette and Klein-MacPhee (2002), possess several limitations that make them inadequate for current applications.

Primarily, these models are constructed solely on observations of alewives' daytime behavior, neglecting their significant nocturnal activity patterns.
Recent studies have revealed that alewives are substantially active at night, engaging in feeding and exhibiting substantial downstream movement during these nocturnal periods (Janssen 1978; Janssen and Brandt 1980; Greene et al. 2009; McCartin et al. 2019).
Collette and Klein-MacPhee (2002) and Greene et al. (2009) even note that groups of alewives spawn in the evening. Consequently, the primary focus on daytime behavior in the existing models fails to capture the true habitat preferences and requirements of alewives, particularly in estuary and brackish environments.

Current models predominantly consider variables such as temperature, depth, and substrate, while disregarding other crucial factors that significantly influence alewives' habitat selection, including flow velocity, sub-aquatic vegetation, and life stage differences. Stevens, Saunders, and Duffy (2021) suggests that considerable diversity in these life history patterns may exist and that life cycle diversity may be an under-examined aspect of the ecology and management of river herring. This limited scope results in incomplete assessments of habitat suitability. Further, existing models fall short of encompassing the total knowledge available for alewives, as inconsistencies and potential inaccuracies emerge from conflicting information concerning substrate, salinity, and depth preferences. Further, current modeling practice lacks transparency and thorough testing, exhibiting concerns such as ad hoc model design, unknown sensitivities, uncertainties in predictions, unclear parameterization sources, inappropriate application domains, limited understanding of model behavior, and insufficient analysis (Schmolke et al., 2010). These limitations undermine the models' effectiveness in predicting habitat suitability for alewives, and since the release of these models, updated observations and stock assessments have been published that offer more detailed information on the habitat for alewives.

To address these shortcomings, habitat models should encompass a more comprehensive understanding of river herring behavior, specifically acknowledging their use of estuarine and brackish habitats. These habitats serve as critical areas for alewives, exhibiting relatively high levels of habitat use (Greene et al. 2009; McCartin et al. 2019; Stevens, Saunders, and Duffy 2021). Incorporating these estuarine and brackish areas into management strategies is important to ensure the conservation and successful management of the species.

The importance of updating habitat models for river herring cannot be overstated. The last comprehensive habitat model for alewives and blueback herring was published in 1983, and since then, a wealth of new biological information has emerged. These developments include significant observations and a deeper understanding of river herring biology, urging the need for updated habitat models to accurately reflect these advancements. Recognizing the distinct life stage differences in river herring is crucial for effective management. Separate models for each species and life stage, considering temporal and spatial variations in larval growth and mortality (\textbf{overton\_ecology\_2012?}), emphasize the necessity for species-specific management plans (\textbf{schmidt\_population\_2003?}). River herring exhibit diverse habitat requirements at different life stages, emphasizing the need for adaptive management approaches. Updating habitat models based on this knowledge allows for a comprehensive understanding of the habitat resources essential for healthy development at various life stages. This expanded insight enables the incorporation of critical resources into management strategies, contributing to more effective and informed approaches for the conservation and sustainable management of river herring populations.

\hypertarget{research-objectives}{%
\section{Research Objectives}\label{research-objectives}}

The main goal of this report is to create an improved habitat model tailored for the spawning adult, non-migratory juvenile, and egg/larvae life stages of alewives and Blueback herring. By customizing habitat preferences for each life stage, the aim is to provide more effective tools for river herring management. A crucial aspect of this effort is to integrate the latest biological information into the habitat models, including recent observations and experimental findings. Recognizing the importance of considering life stage differences in river herring, the research emphasizes the need to develop separate habitat models for each life stage of both river herring species. In summary, the overall objective of this report is to present an updated habitat model that comprehensively captures the habitat preferences of spawning adults, non-migratory juveniles, and egg/larvae life stages for both alewives and Blueback herring.

This report is structured to systematically address the research objectives. It includes an introduction, model development stages, separate chapters on Alewife and Blueback herring models, the application, evaluation, discussion, and a summary. This organization ensures a thorough exploration and communication of updated habitat models for effective river herring management.

\hypertarget{method}{%
\chapter{Model Development}\label{method}}

Ecological modeling serves as a unique tool to understand complex species and ecosystems, providing insights into habitat dynamics and species interactions. There are five steps to ecological modeling including: conceptualization, quantification, application, evaluation, and communication (Grant \& Swannack, 2008). Each stage plays a pivotal role in the development and utilization of habitat suitability models for river herring, facilitating a comprehensive exploration of their habitat preferences, life stages, and responses to environmental variables. This chapter describes the systematic approach undertaken to conceptualize, quantify, apply, evaluate, and communicate the habitat suitability models, thereby contributing to a nuanced understanding of river herring ecology.

\hypertarget{conceptualization}{%
\section{Conceptualization}\label{conceptualization}}

\begin{itemize}
\item
  Introduction to conceptual figure.
\item
  Explanation of why Habitat Suitability Index (HSI) was chosen: To uncover unknown habitat and update preferences based on new observations (last published habitat model for river herring in 1983 by pardue et al). Brief discussion on strengths and weaknesses with HSI and their controversy, over-use in ecology.
\item
  Discussion of chosen parameters (temperature, depth, salinity, flow velocity, substrate) and their significance to fish habitat and habitat selection.
\end{itemize}

\hypertarget{quantification}{%
\section{Quantification}\label{quantification}}

\begin{itemize}
\tightlist
\item
  Explanation of separate models for each species and life stage based on temporal and spatial differences in larval growth and mortality (Overton et al., 2012).
\item
  Emphasis on the need for species-specific management plans due to differences in biology (Schmidt et al., 2003).
\item
  Derivation of HSI breakpoints based on literature, documented observations, and in-situ laboratory experiments. Highlighting the sensitivity of salinities and flow velocities in river herring development.
\item
  Equal weighting (1.5X) given to temperature and flow velocity, identified as more influential on habitat preference (weighted mean HSI).
  There are 6 models all together 1 model for eggs \& larvae, non-migratory juveniles, spawning adults for each species of river herring (Alewives and Blueback Herring).
\end{itemize}

\hypertarget{application}{%
\section{Application}\label{application}}

In the application phase, the models developed for Eggs \& Larvae, Non-Migratory Juvenile, and Spawning Adults of both Alewives and Blueback Herring are put to practical use. This involves a comprehensive demonstration of the model's utility through the presentation of results for each specific life stage and species. The application showcases how the models can predict habitat suitability based on key parameters such as temperature, depth, salinity, flow velocity, and substrate. By providing detailed insights into the habitat preferences and suitability values for different life stages, the application phase establishes a foundation for the subsequent model evaluation process.

\hypertarget{evaluation}{%
\section{Evaluation}\label{evaluation}}

Model evaluation process encompasses three key aspects: System Quality, Technical Quality, and Usability, aligning with the framework established in \emph{NYBEM Report 2023}. System Quality involves a qualitative language assessment to evaluate the sensibility of model parameters and their relevance to river herring habitat. Technical Quality entails validating the model's functionality and performance through an assessment of results obtained from the application. This involves a sensitivity analysis to understand the suitability values indicating herring habitat, a comparison with identified habitat against historic herring habitat maps, and consideration of the quality of input data. Usability delves into the variety of data that can be utilized as inputs into the model, the difficulty of pre-processing, a discussion of output types, and the complexity of post-processing tasks. This evaluation strategy ensures a robust assessment of the ecological models' effectiveness, reliability, and relevance to river herring ecology.

\hypertarget{communication}{%
\section{Communication}\label{communication}}

The intended audience for these models encompasses a broad spectrum of fisheries and aquatic management stakeholders, community members, researchers, and river herring enthusiasts. Recognizing potential variations in modeling knowledge within this diverse audience, model communication was tailored to ensure accessibility and comprehension.

A multifaceted approach was employed to achieve this goal. Clear and informative figures are utilized to visually represent complex ecological dynamics, providing stakeholders with a tangible and intuitive grasp of the model outcomes. Additionally, concise tables of model breakpoints are presented, offering a summary of critical habitat suitability values.

As part of our communication strategy, we use multiple forms of evaluation to clearly demonstrate the reliability and robustness of the models. This involves integrating qualitative assessments, sensitivity analyses, and comparisons with historical habitat maps. The objective is to convey a comprehensive evaluation framework that resonates with a broad audience.

Transparent communication of model limitations and assumptions is a top priority in model communication. Clear communication is crucial to prevent confusion in management applications. Through explicit discussions of the constraints and assumptions inherent in the models, along with clear documentation of the application steps, the aim is to facilitate informed decision-making and encourage constructive engagement with the ecological models. The model communication strategy is designed to bridge the gap between complex ecological modeling and diverse stakeholders to foster a shared understanding of river herring habitat dynamics and support informed and collaborative herring management practices.

\hypertarget{alewife}{%
\chapter{\texorpdfstring{Alewife (\emph{Alosa pseudoharengus})}{Alewife (Alosa pseudoharengus)}}\label{alewife}}

This chapter is dedicated to the habitat preferences and life cycle of alewives (\emph{Alosa pseudoharengus}) for the Northeastern United States. Despite their historical significance, alewife populations have encountered significant declines, leading to their classification as a ``species of concern'' by the U.S. National Marine Fisheries Service (National Marine Fisheries Service 2009). Various factors contribute to this decline, including deteriorating water quality, habitat loss, offshore bycatch/overfishing, increased predation, and dam construction (Kocovsky et al. 2008; National Marine Fisheries Service 2009; Bethoney, Stokesbury, and Cadrin 2014; Waldman and Quinn 2022). The consideration for inclusion in the U.S. Endangered Species List has also been raised, as indicated in reports by the National Marine Fisheries Service in 2013 (National Marine Fisheries Service 2013).

Recent stock assessments reveal diverse trends in documented alewife runs over the last ten years, with some populations showing signs of stabilization or even growth (ASMFC 2017). In 2019, the National Marine Fisheries Service concluded that listing the alewife as threatened or endangered under the Endangered Species Act (ESA) was not warranted (National Marine Fisheries Service 2019).

Alewives are widely distributed throughout the Northeastern United States, thriving in freshwater rivers and estuaries along the Atlantic coast (ASMFC 1985). While historically known for extensive migrations to spawn in freshwater tidal systems, limited information is available about estuary movements for alewives. This chapter aims to explore the favorable habitat conditions for spawning alewife adults, non-migratory juveniles, as well as eggs and larvae, considering factors such as temperature, depth, salinity, flow velocity, and substrate.

\hypertarget{life-cycle-overview}{%
\section{Life cycle overview}\label{life-cycle-overview}}

Alewives have a complex life cycle with distinct stages and behaviors. Spawning occurs in waves during spring, triggered by rising water temperatures and increasing day length (ASMFC 2009; McCartin et al. 2019; Able et al. 2020). Adult alewives migrate upstream from marine environments to suitable brackish or freshwater spawning habitats (Pardue 1983; Collette and Klein-MacPhee 2002). Recent observations also suggest a correlation between alewife migration and the lunar phase (Legett et al. 2021).

Alewife migration and spawning precedes blueback herring by 2-3 weeks (Fay, Neves, and Pardue 1983). Alewives exhibit a north-south seasonal migration pattern, with migrating adults potentially detouring into estuaries (Greene et al. 2009). Upon arrival at the spawning grounds, adult alewives engage in immense spawning runs, where large groups gather to deposit their adhesive eggs over a variety of substrates (O'Connell and Angermeier 1997; Able et al. 2020). After spawning, both males and females return to the marine environment.

In the spawning habitat, the incubation period for eggs lasts 3-6 days (Munroe 2000; Collette and Klein-MacPhee 2002). The eggs hatch into yolk-sac larvae, representing an early developmental stage where larvae rely on an attached yolk sac for nutrients before transitioning to external feeding. For alewives, this stage typically lasts 2-5 days (Bourne 1990).

After this stage, larvae remain in the estuary to grow and migrate downstream towards more brackish areas, maturing into juvenile fish (Pardue 1983). These brackish areas serve as nurseries until the juvenile alewives migrate to the sea (Laney 1997; Kosa and Mather 2001). Alewife larvae can adapt to high salinity (35 psu) by 50 days post-hatch (DiMaggio et al. 2016). However, the survival rate for larvae is relatively low, with only a small percentage successfully reaching the sea, potentially as low as 1\%, depending on ecosystem conditions (Kissil 1974). Similarly, mortality rates for migratory adults during a spawning season can reach as high as 90\% (Brady et al. 2005).

\hypertarget{spawning-adult-alewives}{%
\section{Spawning Adult Alewives}\label{spawning-adult-alewives}}

\hypertarget{habitat-preferences}{%
\subsection{Habitat Preferences}\label{habitat-preferences}}

Spawning adult alewives exhibit specific habitat preferences and requirements. Their annual migration during spawning is energetically demanding, with notable variations in behavior. Some studies suggest fasting during the day and extensive feeding at night, while others report refraining from eating until their return downstream to productive tidal habitats (Janssen and Brandt 1980; Collette and Klein-MacPhee 2002; Greene et al. 2009).

Preferred spawning habitats include lacustrine and fluvial environments rather than riverine ones (Reback et al. 2004; Frank et al. 2011). Alewives spawn and rear in upper river pools (Turner and Limburg 2016; Greene et al. 2009).

\hypertarget{temperature}{%
\subsubsection{Temperature}\label{temperature}}

Spawning adult alewives exhibit distinct temperature preferences crucial to their reproductive success and migration patterns. The spawning process commences at temperatures around 10.5 °C, with upstream migration initiating within the range of 5°C to 10°C (Cianci 1969; Mullen, Fay, and Moring 1986; Loesch 1987; Munroe 2000; Reback et al. 2004). Notably, limited in-stream movement is observed below 8°C or over 18°C, emphasizing the species' preference for moderate temperature conditions (Durbin, Nixon, and Oviatt 1979; Collette and Klein-MacPhee 2002). Spawning ceases at temperatures exceeding 27°C, indicating an upper limit for optimal reproductive activity (Kissil 1974; Pardue 1983; Brown et al. 2000; Reback et al. 2004). The optimal temperature range for successful spawning is broadly identified as 10°C to 22°C, with consensus suggesting peak spawning occurring within the narrower range of 12 to 16 degrees Celsius (Tyus 1974; Pardue 1983; O'Connell and Angermeier 1997; Collette and Klein-MacPhee 2002). While deviations from the optimal temperature range can impact spawning success and migration timing, the overall consensus indicates that alewives thrive in moderate temperature conditions.

\hypertarget{depth}{%
\subsubsection{Depth}\label{depth}}

Spawning adult alewives exhibit specific depth preferences during their reproductive activities. Notably, adult alewives migrating inland show an overall preference for depths below 100 meters (Pardue 1983; Munroe 2000). Species of herring from the same clupeid family have been known to spawn at depths ranging from 0.5 to 15 meters (Haegele and Schweigert 1985). Alewives generally favor depths from Mean Low Tide (MLT) to 10 meters (Brown et al. 2000). Other studies report that alewives favor more shallow spawning habitats ranging from 0.15 to 3 meters (Pardue 1983; Greene et al. 2009), with significant spawning recorded around 0.5 meters (Kosa and Mather 2001). Additional field observations, reveal a significant proportion of alewives occupying habitats as shallow as 2 meters (Mather et al. 2012), with the majority of spawning occurring at less than 1 m (Murdy, Birdsong, and Musick 1997). This variability in depth preferences reflects the adaptability of alewives to shallow conditions during their spawning and migration phases.

\hypertarget{salinity}{%
\subsubsection{Salinity}\label{salinity}}

The documented behavior of alewives challenges conventional beliefs about anadromous species exclusively relying on freshwater environments for spawning. Observations reveal that alewives engage in spawning activities in freshwater tidal habitats with minimal salinity concentrations, demonstrating adaptability to environments with salinity levels below 0.5 psu (Pardue 1983; Brown et al. 2000). This adaptability extends to tolerating higher salinity levels, with successful spawning observed in concentrations of 8 psu (Able et al. 2020). Other research by Brown et al. (2000) emphasizes a heightened preference for habitats with salinity concentrations below 15 psu, deeming concentrations exceeding 20 psu unsuitable for spawning adults.

Field studies further document that adult alewives exhibit spawning activities across diverse estuary habitats with varying salinity levels, including coastal ponds, pond-like regions in coastal rivers and streams, oxbows, eddies, backwaters, stream pools, and flooded swamps (Pardue 1983; Mullen, Fay, and Moring 1986; Collette and Klein-MacPhee 2002; Walsh, Settle, and Peters 2005). Remarkably, species of herring from the same clupeid family have been known to spawn in a wide range of salinity, from freshwater to levels exceeding 35 psu (Haegele and Schweigert 1985). Notably, utilizing estuaries and brackish habitats for spawning may offer energetically favorable conditions for alewives, as it eliminates the need for them to acclimate to complete freshwater environments (DiMaggio et al. 2015).

\hypertarget{flow-velocity}{%
\subsubsection{Flow Velocity}\label{flow-velocity}}

The flow velocity preferences for adult alewives are crucial for understanding spawning habitat (Tommasi et al. 2015). Haro et al. (2004) highlighted the challenges faced by smaller species, including alewives, when subjected to high velocities above 3.5 m/s, leading to impingement issues and poor performance. In contrast, lower velocities, particularly below 1.7 m/s, as observed by Walsh, Settle, and Peters (2005), are favored by alewives. Consistently, Mather et al. (2012) found that tagged alewives spent minimal time in riffle--run habitats but preferred pools, with varying locations of pool occupancy.

While conventional wisdom suggests that alewives spawn in slow-moving habitats with little to no current, Haro et al. (2004) swimming experiments also revealed that migratory alewives can travel farther distances upstream when flow velocities are increased to 1.5 m/s, compared to 3.5 m/s. However, these experiments indicated very little suitability for upstream migration when velocities reach above 3.5 m/s, suggesting no suitability if tested at 4.5 m/s (Haro et al. 2004). Older findings identify velocities up to 0.3 m/s as optimal for spawning (Pardue 1983). Understanding the influence of flow velocity is key for effective management and preservation of the habitat conditions required for successful alewife spawning.

\hypertarget{substrate}{%
\subsubsection{Substrate}\label{substrate}}

Previous studies have presented conflicting information regarding the substrate preferences of spawning adult alewives, often stemming from the generalization of alewives with blueback herring as river herring. Despite Brown et al. (2000) arguing that substrate composition holds no significance in alewife models, contrasting observations from various sources, including studies by Fay, Neves, and Pardue (1983), Killgore, Morgan, and Hurley (1988), O'Connell and Angermeier (1997), and Able et al. (2020), present compelling evidence of a more defined range of substrate preferences. These preferences span a variety of unconsolidated substrates, such as small gravel, pebbles, small cobbles, sand, detritus, and other softer substrates (Esdall 1964; Pardue 1983; Bourne 1990; O'Connell and Angermeier 1997; Greene et al. 2009). Observations from Boger (2002) found that river herring spawning areas along the Rappahannock River,
Virginia, had substrates that consisted of sand, pebbles, and cobbles, and little accumulation of vegetation and detritus. Further, there is an indication that spawning adult alewives may avoid habitats containing sub-aquatic vegetation (SAV), according to findings by Killgore, Morgan, and Hurley (1988), Laney (1997) and Greene et al. (2009). This suggests that spawning adult alewives may favor habitat without SAV.

\hypertarget{habitat-suitability-model}{%
\subsection{Habitat Suitability Model}\label{habitat-suitability-model}}

The updated Habitat Suitability Index (HSI) model for alewives introduces several environmental factors that relate to habitat preferences for spawning adult alewives. In terms of temperature preferences, the model introduces specific temperature ranges in which spawning behavior can be expected, such as the optimal suitability range of 12 to 16 degrees Celsius, offering a more precise depiction of alewives' thermal requirements. The upper limit of 27 degrees Celsius is also emphasized, underlining the critical need for suitable thermal conditions in spawning habitats.

The depth preferences in the model offer a more detailed assessment of habitat suitability, specifying a peak range of 0 to 2 meters for spawning adult alewives. Beyond 20 meters, the model identifies unsuitable depths for spawning, contributing to a clearer view of alewives' depth requirements. Similarly, the updated salinity preferences are characterized by a broader range, with the highest suitability observed at 0 to 8 psu, and moderate suitability expected from 8 to 15 psu.

In the realm of flow velocity, the model introduces distinct ranges, highlighting the optimal conditions at 0 to 0.3 meters per second and extending high suitability to 0.3 to 1.7 meters per second. An upper threshold of 4.5 meters per second is defined, aligning with observed behaviors and contributing to a more detailed view of the flow velocity requirements for alewives. The model also emphasizes the importance of diverse substrate types, including both soft (e.g., small gravel, sand, silt, detritus) and hard substrates (e.g., rock, boulders, clam beds), as well as the consideration of SAV presence or absence in spawning habitats. This update enhances the understanding of alewife spawning habitat and provides valuable insights for effective management and conservation efforts.

\begin{longtable}[]{@{}
  >{\raggedright\arraybackslash}p{(\columnwidth - 4\tabcolsep) * \real{0.3452}}
  >{\raggedright\arraybackslash}p{(\columnwidth - 4\tabcolsep) * \real{0.2619}}
  >{\raggedright\arraybackslash}p{(\columnwidth - 4\tabcolsep) * \real{0.3810}}@{}}
\caption{Model Parameters and Habitat Suitability Values for Spawning Adult Alewives}\tabularnewline
\toprule\noalign{}
\begin{minipage}[b]{\linewidth}\raggedright
\textbf{Parameter}
\end{minipage} & \begin{minipage}[b]{\linewidth}\raggedright
Range
\end{minipage} & \begin{minipage}[b]{\linewidth}\raggedright
\textbf{Habitat Suitability Value}
\end{minipage} \\
\midrule\noalign{}
\endfirsthead
\toprule\noalign{}
\begin{minipage}[b]{\linewidth}\raggedright
\textbf{Parameter}
\end{minipage} & \begin{minipage}[b]{\linewidth}\raggedright
Range
\end{minipage} & \begin{minipage}[b]{\linewidth}\raggedright
\textbf{Habitat Suitability Value}
\end{minipage} \\
\midrule\noalign{}
\endhead
\bottomrule\noalign{}
\endlastfoot
A. Temperature (°C) & \(0.0 < t < 3.0\)

\(3.0 <= t < 5.0\)

\(5.0 <= t < 8.0\)

\(8.0 <= t < 10.5\)

\(10.5 <= t < 12\)

\(12 <= t < 16\)

\(16 <= t < 22\)

\(22 <= t < 27\)

\(t >= 27\) & 0.0

0.1

0.5

0.7

0.8

1.0

0.8

0.5

0.0 \\
B. Depth (\emph{meters}) & \(0.0 < d < 2.0\)

\(2.0 <= d < 5.0\)

\(5.0 <= d < 10.0\)

\(10.0 <= d < 15.0\)

\(15.0 <= d <= 20.0\)

\(d > 20.0\) & 1.0

0.8

0.5

0.3

0.1

0.0 \\
C. Salinity (\emph{psu}) & \(0.5 <= s < 8.0\)

\(8.0 <= s < 15.0\)

\(15.0 <= s <= 20.0\)

\(s > 20\) & 1.0

0.5

0.3

0.0 \\
D. Flow Velocity (\emph{m/s}) & \(0 < v < 0.3\)

\(0.3 <= v < 1.7\)

\(1.7 <= v < 3.5\)

\(3.5 <= v < 4.5\)

\(v >= 4.5\) & 1.0

0.8

0.5

0.1

0.0 \\
E. Substrate & Hard Substrate

Soft Substrate

Present SAV

Absent SAV & 0.3

1.0

0.1

1.0 \\
\end{longtable}

\hypertarget{non-migratory-juveniles}{%
\section{Non-Migratory Juveniles}\label{non-migratory-juveniles}}

\hypertarget{habitat-preferences-1}{%
\subsection{Habitat Preferences}\label{habitat-preferences-1}}

Habitat preferences of non-migratory juvenile alewives differ from those of spawning adult alewives, but similar factors, including temperature, depth, salinity, flow velocity, and substrate, impact the abundance and successful development of young alewives, as evidenced in studies conducted by Pardue (1983), Walsh, Settle, and Peters (2005), and Tommasi et al. (2015). A comprehensive understanding of the preferred habitats for juvenile alewives is essential for meeting their ecological needs.

\hypertarget{temperature-1}{%
\subsubsection{Temperature}\label{temperature-1}}

Temperature plays a pivotal role in shaping the distribution, behavior, and early development of non-migratory juvenile alewives (Tommasi et al. 2015). Specific river observations, including the Delaware River at 22°C, Potomac River at 22.3°C, and Nanticoke River at 21.8°C, identify critical temperature thresholds associated with peak juvenile recruitment (Tommasi et al. 2015). The optimal temperature for nursery rearing is found to be 20--23°C (Tommasi et al. 2015), with a some preference for young alewives estimated at 26.3°C (Kellogg 1982). Supporting these findings, observations in the Eel River showed a mean temperature of 23.1°C, while the Herring River exhibited a mean temperature of 25.1°C (Ames and Lichter 2013; Bourne 1990).

The broader suitability range for juvenile recruitment spans 11°C to 28°C {[}Pardue (1983); Fay, Neves, and Pardue (1983); Klauda, Fischer, and Sullivan (1991); Brown et al. (2000); Munroe (2000); Tommasi et al. (2015)). Juvenile alewives do not survive temperatures below 3°C (Otto, Kitchel, and Rice 1976; Kellogg 1982; Pardue 1983; Munroe 2000). Specific thresholds, such as feeding behavior disruption at 6.7°C and schooling behavior disruption at 4.5°C, underscore the vulnerability of juvenile alewives to temperature fluctuations, with complete disorientation occurring at 2.8°C (Mullen, Fay, and Moring 1986). Exposure to temperature extremes outside the suitable range can negatively impact juvenile growth and performance (Kellogg 1982; Henderson and Brown Jr. 1985; Pörtner and Farrell 2008; Overton, Jones, and Rulifson 2012). Maintaining temperatures within the suitable thresholds is crucial for ensuring successful development and overall health of non-migratory juvenile alewives as they transition to adulthood.

\hypertarget{depth-1}{%
\subsubsection{Depth}\label{depth-1}}

The depth preferences of non-migratory juvenile alewives, as observed in studies such as Brown et al. (2000) and Mullen, Fay, and Moring (1986), differ significantly from their adult counterparts. Juveniles exhibit a preference for depths ranging from 0 to 10 meters, and no observed habitat suitability beyond 20 meters (Brown et al. 2000; Höök et al. 2008). Additional research by Pardue (1983) reinforces this trend, indicating that juveniles favor depths between 0.5 to 5 meters, and their abundance notably increases around five-meter depths. Similarly, other juvenile cluepids were found abundant in 4.6 meters (Mullen, Fay, and Moring 1986). After leaving the estuary, juveniles can be found in deeper waters off-shore before swimming farther out to sea (Greene et al. 2009). These combined field observations show that the optimal depth for juvenile alewives is equal to or less than 5 meters.

\hypertarget{salinity-1}{%
\subsubsection{Salinity}\label{salinity-1}}

Juvenile alewives display considerable adaptability to varying salinity levels, as evidenced by physiological changes observed in response to seawater exposure. This adaptation suggests capacity for osmoregulation, enabling them to regulate salt and water balance and maintain stability in seawater environments (Christensen et al. 2012). Additional studies, such as Leim (1924), Pardue (1983), and Brown et al. (2000), provide further insights into the versatility of alewives in diverse salinity conditions and imply that alewives can thrive in environments with fluctuating salinity, emphasizing adaptability to both freshwater and saltwater habitats during different life stages.

Research by Richkus (1975) indicates that juvenile alewives, transferred between freshwater and saline water, experienced zero mortality, suggesting a robust tolerance to salinity changes. However, DiMaggio et al. (2016) note lower survival rates when transferred from higher salinities to lower ones, emphasizing a potential preference for higher salinity environments. The presence of juvenile alewives in estuarine waters, where they are preyed upon by bluefish (Creaser and Perkins 1994), underscores their distinct salinity preference, spanning from areas exceeding 10 psu to levels as high as 30 psu (Pardue 1983; Brown et al. 2000). Turner and Limburg (2016) and Able et al. (2020) highlight juvenile preference for estuarine habitats with salinity concentrations ranging from 0.5 to 25 psu, promoting an ideal balance between freshwater and marine conditions for growth. Overall, juvenile alewives exhibit a wide range salinity preferences, showcasing their adaptability to diverse environments for optimal growth and survival.

\hypertarget{flow-velocity-1}{%
\subsubsection{Flow Velocity}\label{flow-velocity-1}}

Flow velocity preferences play a pivotal role in the development and survival of non-migratory juvenile alewives. Laboratory experiments suggest that juvenile alewives tend to avoid water velocities exceeding 0.1 m/s (Greene et al. 2009). Other observations document the presence of juvenile alewives in habitats with flow velocities ranging from 0.05 to 0.17 m/s (Richkus 1975; O'Connell and Angermeier 1999). Slower flow rates are conducive to energy conservation and effective foraging, while higher flow velocities may impede access to critical food resources, and disrupt their position in the water column (Haro et al. 2004; Able et al. 2020). Understanding flow velocity preferences is essential for informed habitat management, ensuring successful transitions for non-migratory juvenile alewives to adulthood.

\hypertarget{substrate-1}{%
\subsubsection{Substrate}\label{substrate-1}}

Non-migratory juvenile alewives have broad substrate preferences, underscoring their adaptability to diverse aquatic environments. While earlier studies, such as Fay, Neves, and Pardue (1983), indicated a preference for sandy substrates, recent observations suggest a potential inclination towards rocky substrates (Janssen and Luebke 2004; Kornis and Janssen 2011; Boscarino et al. 2020). Alewife catches at rocky sites have been reported to be much higher than those at sandy sites, possibly attributed to the increased profitability of rocky habitats for feeding on zooplankton (Janssen and Luebke 2004; Kornis and Janssen 2011). These substrate preferences have implications for habitat management,and creating environments conducive to the successful development and survival of juvenile alewives.

The presence of seagrass and SAV is important to non-migratory juvenile alewife habitat. Despite some studies suggesting avoidance of areas with aquatic vegetation (Ingel 2013), research by Olney and Boehlert (1988), Laney (1997), Greene et al. (2009), and Smith and Rulifson (2015) contradict this, emphasizing that seagrass beds provide essential nursery habitats for juvenile alewives. These vegetated areas serve as refuge from predators and offer abundant food sources. Seagrass beds also contribute to enhanced water quality by stabilizing sediments and promoting nutrient cycling, creating an optimal environment for the thriving of juvenile alewives. The significance of SAV extends to overwintering habitats, as suggested by Killgore, Morgan, and Hurley (1988). The understanding of these diverse substrate preferences and seagrass coverage is vital for effective habitat management, ensuring the successful development of non-migratory juvenile alewives in varied aquatic environments.

\hypertarget{habitat-suitability-model-1}{%
\subsection{Habitat Suitability Model}\label{habitat-suitability-model-1}}

The updated Habitat Suitability Index (HSI) model for non-migratory juvenile alewives takes into account temperature, depth, salinity, flow velocity, and substrate, which influence habitat availability. The model defines habitat preferences for non-migratory juveniles using these parameters to provide an understanding of suitable conditions for the development and survival of non-migratory juveniles.

Temperature preferences are detailed, highlighting the critical range of 20 to 23 degrees Celsius as optimal for alewives' habitat. Depth considerations underscore the importance of depths below 5.0 meters, with higher depths exceeding 20.0 meters identified as unsuitable. The model recognizes the adaptability of alewives to varying salinity conditions, emphasizing their high suitability in habitats ranging from 0.5 to 25.0 psu.

A specific range is introduced for flow velocity to address optimal conditions for juvenile habitat. The model provides valuable insights into the impact of flow velocity on survival during development, helping inform habitat management practices. The substrate parameter delves into substrate type and the presence of SAV. Hard substrate receives the highest suitability value, while the presence of SAV contributes positively to habitat suitability. This detailed substrate analysis enhances our understanding of alewives' preferences for specific environmental conditions.

In summary, the HSI model offers a detailed perspective on habitat preferences for non-migratory juvenile alewives, providing detailed ranges and suitability values for each parameter. This approach contributes to comprehension of the conditions essential for the successful development and survival of alewives in their aquatic habitats.

\begin{longtable}[]{@{}
  >{\raggedright\arraybackslash}p{(\columnwidth - 4\tabcolsep) * \real{0.3452}}
  >{\raggedright\arraybackslash}p{(\columnwidth - 4\tabcolsep) * \real{0.2619}}
  >{\raggedright\arraybackslash}p{(\columnwidth - 4\tabcolsep) * \real{0.3810}}@{}}
\caption{Model Parameters and Habitat Suitability Values for Non-Migratory Juvenile Alewives}\tabularnewline
\toprule\noalign{}
\begin{minipage}[b]{\linewidth}\raggedright
\textbf{Parameter}
\end{minipage} & \begin{minipage}[b]{\linewidth}\raggedright
Range
\end{minipage} & \begin{minipage}[b]{\linewidth}\raggedright
\textbf{Habitat Suitability Value}
\end{minipage} \\
\midrule\noalign{}
\endfirsthead
\toprule\noalign{}
\begin{minipage}[b]{\linewidth}\raggedright
\textbf{Parameter}
\end{minipage} & \begin{minipage}[b]{\linewidth}\raggedright
Range
\end{minipage} & \begin{minipage}[b]{\linewidth}\raggedright
\textbf{Habitat Suitability Value}
\end{minipage} \\
\midrule\noalign{}
\endhead
\bottomrule\noalign{}
\endlastfoot
A. Temperature (°C) & \(0 > t < 3\)

\(3 <= t < 7\)

\(7 <= t < 11\)

\(11 <= t < 20\)

\(20 <= t < 23\)

\(23 <= t < 28\)

\(t >= 28\) & 0.0

0.1

0.5

0.8

1.0

0.5

0.1 \\
B. Depth (\emph{meters}) & \(0 < d < 5.0\)

\(5.0 <= d < 10.0\)

\(10.0 <= d < 20.0\)

\(d >= 20\) & 1.0

0.7

0.5

0.0 \\
C. Salinity (\emph{psu}) & \(0 < s < 0.5\)

\(0.5 <= s < 10.0\)

\(10.0 <= s <= 25.0\)

\(s > 25\) & 0.0

0.5

1.0

0.8 \\
D. Flow Velocity (\emph{m/s}) & \(0 < v < 0.1\)

\(0.1 <= v < 0.17\)

\(0.17 <= v < 0.3\)

\(0.3 <= v < 3.5\)

\(3.5 <= v < 4.5\)

\(v >= 4.5\) & 1.0

0.7

0.5

0.3

0.1

0.0 \\
E. Substrate & Hard Substrate

Soft Substrate

Present SAV

Absent SAV & 1.0

0.1

1.0

0.5 \\
\end{longtable}

\hypertarget{eggs-larvae}{%
\section{Eggs \& Larvae}\label{eggs-larvae}}

\hypertarget{habitat-preferences-2}{%
\subsection{Habitat Preferences}\label{habitat-preferences-2}}

The early life stages of alewives, particularly their planktonic larval phases, are characterized by vulnerability to passive transport dispersal (Schmidt and Kiviat 1988). Understanding the habitat preferences exhibited by alewife eggs and larvae is essential for comprehending their developmental dynamics and ensuring optimal conditions for survival. This section will delve into key environmental parameters influencing the distribution and behavior of alewife eggs and larvae. Factors such as temperature, depth, salinity, flow velocity, and substrate play crucial roles in shaping the habitat preferences of these early life stages. Notably, studies, such as those by O'Connell and Angermeier (1997), emphasize current velocity as one of the strongest predictors for the presence of alewife eggs.

\hypertarget{temperature-2}{%
\subsubsection{Temperature}\label{temperature-2}}

Water temperature significantly shapes the growth, survival, and overall development of alewife eggs and larvae. Pardue (1983) and Kellogg (1982) reveal the optimal temperature for larval growth at approximately 26.4°C, with the estimated peak at 26.3°C. A broad suitable temperature range is evident as egg survival occurs between 12°C and 30°C, and temperatures exceeding 31°C lead to larvae mortality. Beyond direct developmental impacts, temperature influences the suitability of nursery areas, as noted by Tommasi et al. (2015). Specific river systems, like the Chowan River, experience temperatures surpassing the upper thermal tolerance of alewife larvae, influencing their distribution (Kellogg 1982).

The association between peak densities of alewife larvae and water temperature peaks highlights temperature sensitivity in early life stages of alewife. Lower temperatures and peaks in river flow negatively affect alewife larvae, indicating a complex interplay between temperature and environmental conditions (O'Connell and Angermeier 1999). Optimal temperature ranges for egg density peaks fall between 16-21°C and 11-28°C, showcasing the adaptability of alewife eggs to varying temperature conditions (O'Connell and Angermeier 1999; Klauda, Fischer, and Sullivan 1991). This is supported in observations from Kellogg (1982) where maximum hatching success occurred at 12.7°C, 15.0°C, and 20.8°C. Optimal hatching temperature is defined at 16°C (Esdall 1970), with hatching success ceasing above 29.7°C (Kellogg 1982; Pardue 1983). Temperature is a crucial factor impacting the development and survival of alewife eggs and larvae.

\hypertarget{depth-2}{%
\subsubsection{Depth}\label{depth-2}}

Alewife eggs and larvae exhibit clear depth preferences, primarily favoring shallow-water habitats during their early life stages. Observations from Lake Ontario, as documented by Klumb et al. (2003) and Ingel (2013), reveal that alewife larvae are uniformly distributed in nearshore areas, with a significant presence in waters less than 5 meters deep. In the same lake, early post-hatch larvae are particularly abundant in depths less than 3 meters, while larger larvae progressively inhabit deeper habitats (Ingel 2013; Dunstall 1984). Studies conducted in Nova Scotia's Margaree River and other locations further emphasize this trend, indicating that alewife larvae predominantly reside in depths shallower than 2 meters (Mullen, Fay, and Moring 1986). The preference for shallow-water habitats is linked to the benefits these environments offer, providing protection from predators and facilitating access to essential food sources, crucial for the growth and development of alewife larvae into juveniles.

\hypertarget{salinity-2}{%
\subsubsection{Salinity}\label{salinity-2}}

While there are observations indicating that alewives utilize freshwater streams for their early life stages (Kosa and Mather 2001; Tommasi et al. 2015), Mullen, Fay, and Moring (1986) provides evidence of their high tolerance to salinity changes. Pardue (1983) reports the presence of alewife eggs and larvae in environments with salinity levels below 12 ppt, suggesting a potential inclination towards slightly saline conditions. This inclination is corroborated by Dovel and Edmunds (1971), who notes that the growth rates of larvae are significantly lower in freshwater compared to slightly saline water, ranging from 1.0 to 1.3 ppt. The adaptability of alewives to a range of salinities is evident in their ability to thrive in different conditions.

Similarly, other clupeid eggs display tolerance to salinities in the range of 3-33\%, and herring eggs and larvae are reported to develop abundantly in seawater with healthy and active larvae (Haegele and Schweigert 1985; Ford 1929). DiMaggio et al. (2016) offer valuable insights into the survival rates of alewives at different salinities, with the highest mean survival rate observed at 15‰ salinity (76.0\%). As salinity increases to 20‰ and 30‰, the mean survival rates for alewives decrease to 69.3\% and 64.7\%, respectively. This pattern suggests a potential preference for moderate salinity conditions, supported by the higher survival rates at 15‰ compared to higher salinities. dimaggio\_effects\_2016 further highlights the adaptability of alewife embryos, showing survival rates above 97\% at low salinities (2, 5, and 10 g/L) and decreased survival at salinities above 15 g/L. Overall, these findings highlight the broad salinity preferences that promote successful early development and survival of alewife eggs and larvae.

\hypertarget{flow-velocity-2}{%
\subsubsection{Flow Velocity}\label{flow-velocity-2}}

The early egg stages of alewives exhibit a positive relationship with flow velocity, particularly in the range of 0.03-0.2 m/s (O'Connell and Angermeier 1999). However, there is a rapid decline in larvae abundance associated with high flows, suggesting that extreme water velocities can be detrimental to their survival (O'Connell and Angermeier 1999). Insights from the Delaware River indicate that excessively high flows may negatively impact alewife recruitment, as elevated water velocity can create barriers that hinder the swimming performance of anadromous fish (Haro et al. 2004).

In specific habitats, larval alewives tend to be consistently found in water velocities up to 0.12 m/sec, avoiding faster currents (Ingel 2013). The distribution of alewife larvae is not primarily influenced by overall tidal fluctuations or the highest speeds of water flow; rather, it is influenced by more localized and instantaneous changes in velocity (Ingel 2013). Previous studies, such as Pardue (1983), observed optimal velocities for larvae and egg development ranging from 0 to 0.3 m/s. These findings collectively highlight the importance of flow velocity in shaping the distribution and development of alewife eggs and larvae, emphasizing the significance of suitable flow conditions for their early life stages.

\hypertarget{substrate-2}{%
\subsubsection{Substrate}\label{substrate-2}}

Substrate preferences significantly influence the distribution and habitat selection of alewife eggs and larvae. The sandy substrate on the eastern side of Lake Michigan was identified as a favorable environment for the primary and secondary production, limiting dreissenid mussel biomass and providing better support for prey available to larval alewife (Prendergast 2019). Detritus concentrations, a key variable associated with larval growth rate, are positively correlated with productive regions in Lake Michigan, indicating increased prey availability for larval alewife in areas with high detritus concentrations (Prendergast 2019). Pardue (1983) also suggests that optimal egg and larval habitat is found in substrates composed of 75\% silt or other soft material containing detritus and vegetation. Soft substrates, such as sand, are considered conducive to the successful development of alewife eggs and larvae. Further, larvae have not been commonly reported in abundance over hard substrates. Reports of eggs on hard substrates might be attributed to the temporary adhesive property of alewife eggs rather than a preference for hard substrate habitats.

Although SAV is thought to provide spawning and nursery habitat for most anadromous and resident fishes, alewife larval catch is inversely related to SAV density, with larvae most frequently found in areas with less than 10 percent vegetation density and never in areas exceeding 30 percent vegetation density (Ingel 2013). Observations by Schmidt and Kiviat (1988) further highlight the impact of vegetation cover, like water-chestnut and water celery beds, on the distribution of larval alewife, with a decline in larval abundance following the establishment of dense water-chestnut cover. These findings collectively highlight the substrate preferences of alewife eggs and larvae, emphasizing the importance of soft substrates and vegetation coverage in their habitat.

\hypertarget{habitat-suitability-model-2}{%
\subsection{Habitat Suitability Model}\label{habitat-suitability-model-2}}

The habitat suitability model for alewife eggs and larvae encompasses specified ranges and corresponding suitability values for temperature, depth, salinity, flow velocity, and substrate. The temperature parameter delineates thermal tolerances, emphasizing an optimal range between 16 and 28 degrees Celsius. Beyond these limits, suitability sharply declines, with extreme temperatures below 3 degrees or above 30 degrees Celsius deemed unsuitable. This highlights the significance of maintaining specific temperature conditions for alewife development to ensure optimal conditions for their survival.

The depth parameter provides insights into the preferred distribution of alewives, with a peak suitability observed in depths ranging from 0 to 3 meters. As depths increase, the suitability progressively decreases, indicating that eggs and larvae are more commonly found in shallower waters. Salinity considerations are also integrated, indicating a higher suitability in habitats with salinity levels between 0.5 and 12 psu. As salinity exceeds 20 psu, the suitability diminishes, suggesting an upper limit for suitable salinity conditions at this life stage in alewives.

The model specifies an optimal range of 0.03 to 0.12 meters per second for flow velocity. This parameter highlights the importance of low flow rates for alewife habitats, while higher velocities result in reduced suitability. The substrate parameter emphasizes the significance of substrate type and the presence of SAV for alewife eggs and larvae. Hard substrates and the absence of SAV are associated with increased abundance, underlining the importance of substrate variety in alewife habitats. Overall, this model defines environmental factors influencing habitat suitability for alewife eggs and larvae, aiding in effective conservation and management strategies.

\begin{longtable}[]{@{}
  >{\raggedright\arraybackslash}p{(\columnwidth - 4\tabcolsep) * \real{0.3452}}
  >{\raggedright\arraybackslash}p{(\columnwidth - 4\tabcolsep) * \real{0.2619}}
  >{\raggedright\arraybackslash}p{(\columnwidth - 4\tabcolsep) * \real{0.3810}}@{}}
\caption{Model Parameters and Habitat Suitability Values for Alewife Larvae and Egg Development Stages}\tabularnewline
\toprule\noalign{}
\begin{minipage}[b]{\linewidth}\raggedright
\textbf{Parameter}
\end{minipage} & \begin{minipage}[b]{\linewidth}\raggedright
Range
\end{minipage} & \begin{minipage}[b]{\linewidth}\raggedright
\textbf{Habitat Suitability Value}
\end{minipage} \\
\midrule\noalign{}
\endfirsthead
\toprule\noalign{}
\begin{minipage}[b]{\linewidth}\raggedright
\textbf{Parameter}
\end{minipage} & \begin{minipage}[b]{\linewidth}\raggedright
Range
\end{minipage} & \begin{minipage}[b]{\linewidth}\raggedright
\textbf{Habitat Suitability Value}
\end{minipage} \\
\midrule\noalign{}
\endhead
\bottomrule\noalign{}
\endlastfoot
A. Temperature (°C) & \(0 < t < 3\)

\(3 <= t < 7\)

\(7 <= t < 11\)

\(11 <= t < 16\)

\(16 <= t < 28\)

\(28 <= t < 30\)

\(t >= 30\) & 0.0

0.1

0.3

0.5

1.0

0.1

0.0 \\
B. Depth (\emph{meters}) & \(0 < d < 3\)

\(3 <= d < 5\)

\(5 <= d < 10\)

\(10 <= d < 20\)

\(d >= 20\) & 1.0

0.8

0.5

0.1

0.0 \\
C. Salinity (\emph{psu}) & \(0 < s < 0.5\)

\(0.5 <= s < 12.0\)

\(12.0 <= s < 15.0\)

\(15.0 <= s <= 20.0\)

\(s > 20\) & 0.8

1.0

0.75

0.7

0.65 \\
D. Flow Velocity (\emph{m/s}) & \(0 < v < 0.03\)

\(0.03 <= v < 0.12\)

\(0.12 <= v < 0.3\)

\(0.3 <= v < 3.5\)

\(3.5 <= v < 4.5\)

\(v >= 4.5\) & 0.7

1.0

0.5

0.3

0.1

0.0 \\
E. Substrate & Hard Substrate

Soft Substrate

Present SAV

Absent SAV & 0.5

1.0

0.1

1.0 \\
\end{longtable}

\hypertarget{blueback}{%
\chapter{\texorpdfstring{Blueback Herring (\emph{Alosa aestivalis})}{Blueback Herring (Alosa aestivalis)}}\label{blueback}}

This chapter aims to explore the habitat preferences and life cycle of blueback herring (\emph{Alosa aestivalis}) in the northeastern United States.
Blueback herring have faced significant declines,

This chapter explores the favorable habitat conditions for spawning blueback adults, non-migratory juveniles, as well as eggs and larvae, which are influenced by factors such as suitable spawning habitats, water quality conditions, and availability of appropriate food resources (Lynch et al. 2015).

blueback herring spawn in higher salinity, faster-moving waters in the lower river (Turner \& Limburg, 2016)

\hypertarget{life-cycle-overview-1}{%
\section{Life cycle overview}\label{life-cycle-overview-1}}

\begin{itemize}
\tightlist
\item
  Alewives enter the river first, followed by blueback herring 2-3 weeks later (Jessop 2003).
\item
  juveniles may take longer to move into brackish waters (Schmidt et al.~1988).
\item
  Blueback larvae demonstrate the ability to adapt to marine environments by 58 days post-hatch (DiMaggio et al., 2015).
\end{itemize}

\hypertarget{habitat-requirements}{%
\section{Habitat Requirements}\label{habitat-requirements}}

\hypertarget{spawning-adults}{%
\subsection{Spawning Adults}\label{spawning-adults}}

\hypertarget{temperature-3}{%
\subsubsection{Temperature}\label{temperature-3}}

\begin{itemize}
\item
  move into estuary begins at 14 and ceases at 27 Pardue (1983)
\item
  16 - 16 klauda et al 1991
\item
  Our results indicate that blueback herring adults require temperatures of at least 16.8''C to spawn. (O'Connell \& Angermeier, 1999)
\item
  range 17 - 26 and optimal 20 - 24 Pardue (1983)
\item
  spawning stops at 27 Pardue (1983)
\item
  Our results demonstrate that the spawning temperature that maximized juvenile blueback herring abundance in the Delaware River was 11 °C (tommasi 2015)
\item
  spawning begins at 14 C (Mullen et al., 1986)
\item
  Alewives begin spawning when water temperatures reach 51 and bluebacks 57 degrees F (Reback et al., 2004).
\item
  Both species cease spawning when the water warms to 81°F (Reback et al., 2004).
\end{itemize}

\hypertarget{depth-3}{%
\subsubsection{Depth}\label{depth-3}}

\begin{itemize}
\tightlist
\item
  0.15--3 m Pardue (1983)
\item
  spawning habitat were all 0.5 m or less (Kosa \& Mather, 2001)
\item
  Species of herring from the same cluepid family have been known to spawn majority 0.5-5m and up to 15 m (Haegele \& Schweigert, 1985).
\item
  Overall depth preferences for spawning blueback seems similar to alewife and therefore same depth preferences are assumed
\end{itemize}

\hypertarget{salinity-3}{%
\subsubsection{Salinity}\label{salinity-3}}

\begin{itemize}
\item
  spawning begins then or shortly thereafter. Although this species apparently spawns primarily in fresh water, spawning also occurs in brackish ponds at Woods Hole (Nichols and Breder, 1926; Hildebrand, 1963).
\item
  I found that the chief spawning grounds in the Delaware River were in tidal water (Chittenden, 1972)
\item
  Species of herring from the same cluepid family have been known to spawn in a range of salinities from freshwater to 35+ psus (Haegele \& Schweigert, 1985).
\item
  Alewives have been widely observed spawning in freshwater tidal habitats with minimal salinity concentrations, revealing a habitat use in environments with salinity levels below 0.5 psu (Pardue 1983).
\item
  Blueback herring in Winnegance Lake and Patten Pond show a preference for high salinity environments for spawning, indicated by their limited residency time in freshwater. The immediate transition of fish from Winnegance Lake to the estuary after spawning suggests a preference for high salinity spawning, with the estuarine environment supporting early egg and larvae development (Payne Wynne et al., 2015). blueback herring seem to be able to tolerate higher salinity levels for spawning activity adn are even known to spawn in areas where freshwater input is extremely minimal
\end{itemize}

\hypertarget{flow-velocity-3}{%
\subsubsection{Flow Velocity}\label{flow-velocity-3}}

\begin{itemize}
\tightlist
\item
  blueback herring show a preferance for higher velocities than alewives (haro 2004)
\item
  prefer fluvial environments which are higher velocity and not stagnant
\item
  Low velocities (0.0--1.7 m/s) (Walsh et al 2005)
\item
  0.11 m/s (O'Connell \& Angermeier, 1997)
\item
  blueback tend to favor running water and a fairly clean hard bottom, Bourne 1990
\item
  Both species spawn primarily in habitats with little or no current Walsh 2005
\end{itemize}

\hypertarget{substrate-3}{%
\subsubsection{Substrate}\label{substrate-3}}

\begin{itemize}
\item
  Blueback herring do not usually spawn as far upstream as the alewife and selectively choose spawning sites in fast-flowing water over a hard substrate, particularly in shared spawning grounds (Loesch and Lund 1977; Jones et al.~1978; Scott and Scott 1988).
\item
  blueback tend to favor running water and a fairly clean hard bottom, bourne 1990
\item
  prefer hard substrate (Loesch and Lund 1977)
\end{itemize}

\hypertarget{non-migratory-juveniles-1}{%
\subsection{Non-Migratory Juveniles}\label{non-migratory-juveniles-1}}

\hypertarget{temperature-4}{%
\subsubsection{Temperature}\label{temperature-4}}

\begin{itemize}
\item
  higher preference than alewife juveniles
\item
  supported in observations where Eel River had a mean temperature of 23.1°C, while the Herring River (Bourne) had temperatures with a mean of 25.1°C (Ames \& Lichter, 2013).
\end{itemize}

\hypertarget{depth-4}{%
\subsubsection{Depth}\label{depth-4}}

\begin{itemize}
\tightlist
\item
  Juvenile Blueback herring in Potomac River, Virginia abundant in 4.6 m(Mullen et al., 1986)
\end{itemize}

\hypertarget{salinity-4}{%
\subsubsection{Salinity}\label{salinity-4}}

Bluebacks:

Bluebacks exhibit high survival rates when transferred between certain salinities, particularly from 3‰, 15‰, and 30‰ into 3‰ salinity.
The high survival rates indicate that Bluebacks are well-adapted and can thrive under a range of salinity conditions within the specified experimental scenarios.
While the information does not explicitly state a preference for specific salinities, the high survival rates suggest that Bluebacks may have a relatively broad salinity tolerance in the studied range.(DiMaggio et al., 2015)

\begin{itemize}
\tightlist
\item
  blueback herring is highly salinity tolerant early in life (28 psu) (Chittenden, 1972)
\item
  herring eggs and larvae developing in abundance in sea-water with healthy and active larvae (Ford, 1929)
\end{itemize}

\hypertarget{flow-velocity-4}{%
\subsubsection{Flow Velocity}\label{flow-velocity-4}}

\begin{itemize}
\tightlist
\item
  Previous optimal velocities for larvae and egg development were observed from 0 to 0.3 m/s (Pardue 1983).
\end{itemize}

\hypertarget{substrate-4}{%
\subsubsection{Substrate}\label{substrate-4}}

\begin{itemize}
\tightlist
\item
  blueback tend to favor running water and a fairly clean hard bottom, bourne 1990
\end{itemize}

\hypertarget{eggs-and-larvae}{%
\subsection{Eggs and Larvae}\label{eggs-and-larvae}}

\hypertarget{temperature-5}{%
\subsubsection{Temperature}\label{temperature-5}}

\begin{itemize}
\tightlist
\item
  blueback herring early egg stages were positively related to water temperature (14-22°C). (O'Connell \& Angermeier, 1999)
\item
  Blueback herring eggs have been found in water temperatures ranging from 7 to 14°C in the upper Chesapeake Bay, with most being collected at 14°C (Dove1 1971).
\item
  Temperatures observed in our study were always below those known to impair hatching success (32.9 to 36.1°C) or cause larval deformities (34°C). (O'Connell \& Angermeier, 1999)
\item
  eggs recorded at 20-24 C (Mullen et al., 1986)
\item
  19 C 0-5\% deformities, 20 0-24\% deformities, Complete larval deformities in eggs exposed to 34 C (Koo and Johnston 1978)
\item
  11.48C to 23.08C, 10.98C to 20.88C walsh 2005
\item
  Larvae were not found in the canal oroxbow until late April and were never abundantat water temperatures below 148C (walsh 2005).
\item
  peak densities were positively associated with peaks in water temperature (within the range of 4-19 C) (O'Connell \& Angermeier, 1999).
\end{itemize}

\hypertarget{depth-5}{%
\subsubsection{Depth}\label{depth-5}}

\begin{itemize}
\tightlist
\item
  In Lake Ontario, early post-hatch larvae were most abundant in shallow areas less than 3m in depth, with larger larvae progressively occupying deeper habitat (Dunstall 1984).
\item
  Larval Alosa in Nova Scotian fmgs rivers inhabite-i-waters that were relatively shallow (\textless2 m) (Mullen et al., 1986)
\end{itemize}

\hypertarget{salinity-5}{%
\subsubsection{Salinity}\label{salinity-5}}

\begin{itemize}
\tightlist
\item
  Bluebacks:
\end{itemize}

The mean survival rates for Bluebacks consistently increase with higher salinities.
The highest mean survival rate for Bluebacks is observed at 30‰ salinity (95.3\%).
This pattern indicates that Bluebacks may exhibit a preference for higher salinity conditions, as indicated by their higher survival rates at 30‰ compared to lower salinities. (DiMaggio et al., 2015)

\begin{itemize}
\item
  highly tolerant of salinity changes (Mullen et al., 1986)
\item
  Pacific herring eggs are tolerant to salinities in the range of 3-33\% (Haegele \& Schweigert, 1985).
\item
  blueback herring embryos displayed a wide salinity tolerance throughout the range. Embryos of both species exhibited high survival in tidal salinity exposures. Survival of acutely-transferred alewife and blueback herring larvae decreased with increasing salinity (\textgreater20 g/L)(DiMaggio et al., 2016).
\end{itemize}

\hypertarget{flow-velocity-5}{%
\subsubsection{Flow Velocity}\label{flow-velocity-5}}

\begin{itemize}
\tightlist
\item
  lower temperatures and peaks in river flow negatively impacted early life stages (O'Connell \& Angermeier, 1999)
\end{itemize}

\hypertarget{substrate-5}{%
\subsubsection{Substrate}\label{substrate-5}}

\begin{itemize}
\tightlist
\item
  thought to provide spawning and nursery habitat for anadromous and resident fishes (Miller et al 2006); however, in the case of alewives, larval catch was inversely related to vegetation density. larvae were most frequently found in areas with less than 10 percent vegetation density and were never found in areas where vegetation density exceeded 30 percent (Ingel, 2013).
\end{itemize}

\hypertarget{habitat-suitability-models}{%
\section{Habitat suitability models}\label{habitat-suitability-models}}

\hypertarget{spawning-adults-1}{%
\subsection{Spawning Adults}\label{spawning-adults-1}}

\begin{longtable}[]{@{}
  >{\raggedright\arraybackslash}p{(\columnwidth - 4\tabcolsep) * \real{0.3494}}
  >{\raggedright\arraybackslash}p{(\columnwidth - 4\tabcolsep) * \real{0.2530}}
  >{\raggedright\arraybackslash}p{(\columnwidth - 4\tabcolsep) * \real{0.3855}}@{}}
\caption{\textbf{Table 1.} Model Parameters and Habitat Suitability Values for Spawning Adult Blueback Herring.}\tabularnewline
\toprule\noalign{}
\begin{minipage}[b]{\linewidth}\raggedright
\textbf{Parameter}
\end{minipage} & \begin{minipage}[b]{\linewidth}\raggedright
Range
\end{minipage} & \begin{minipage}[b]{\linewidth}\raggedright
\textbf{Habitat Suitability Value}
\end{minipage} \\
\midrule\noalign{}
\endfirsthead
\toprule\noalign{}
\begin{minipage}[b]{\linewidth}\raggedright
\textbf{Parameter}
\end{minipage} & \begin{minipage}[b]{\linewidth}\raggedright
Range
\end{minipage} & \begin{minipage}[b]{\linewidth}\raggedright
\textbf{Habitat Suitability Value}
\end{minipage} \\
\midrule\noalign{}
\endhead
\bottomrule\noalign{}
\endlastfoot
A. Temperature (°C) & \(0 < t < 8\)

\(8 <= t < 12\)

\(12 <= t < 16\)

\(16 <= t < 22\)

\(22 <= t < 27\)

\(27 <= t < 30\)

\(t > 30\) & 0.0

0.5

1.0

1.0

0.5

0.1

0.0 \\
B. Depth (\emph{meters}) & \(MLT < d < 2\)

\(2 <= d < 10\)

\(10 <= d < 20\)

\(20 <= d < 50\)

\(50 <= d < 100\)

\(d > 100\) & 0.0

0.5

1.0

0.5

0.5

0.5 \\
C. Salinity (\emph{psu}) & \(0 < s < 0.5\)

\(0.5 <= s < 5.0\)

\(5.0 <= s < 15.0\)

\(15.0 <= s < 20.0\)

\(s > 20\) & 1.0

1.0

1.0

0.5

0.0 \\
D. Flow Velocity (\emph{m/s}) & \(0 < v < 0.3\)

\(0.3 <= v < 1.5\)

\(1.5 <= v < 3.5\)

\(3.5 <= v < 4.5\)

\(v > 5\) & 1.0

1.0

0.5

0.3

0.0 \\
E. Substrate & Hard Substrate

Soft Substrate

Present SAV

Absent SAV & 0.5

1.0

1.0

0.5 \\
\end{longtable}

\hypertarget{temperature-6}{%
\subsubsection{Temperature}\label{temperature-6}}

\hypertarget{depth-6}{%
\subsubsection{Depth}\label{depth-6}}

\hypertarget{salinity-6}{%
\subsubsection{Salinity}\label{salinity-6}}

\hypertarget{flow-velocity-6}{%
\subsubsection{Flow Velocity}\label{flow-velocity-6}}

\hypertarget{substrate-6}{%
\subsubsection{Substrate}\label{substrate-6}}

\hypertarget{non-migratory-juveniles-2}{%
\subsection{Non-Migratory Juveniles}\label{non-migratory-juveniles-2}}

\begin{longtable}[]{@{}
  >{\raggedright\arraybackslash}p{(\columnwidth - 4\tabcolsep) * \real{0.3494}}
  >{\raggedright\arraybackslash}p{(\columnwidth - 4\tabcolsep) * \real{0.2530}}
  >{\raggedright\arraybackslash}p{(\columnwidth - 4\tabcolsep) * \real{0.3855}}@{}}
\caption{\textbf{Table 2.} Model Parameters and Habitat Suitability Values for Non-Migratory Juvenile Blueback Herring.}\tabularnewline
\toprule\noalign{}
\begin{minipage}[b]{\linewidth}\raggedright
\textbf{Parameter}
\end{minipage} & \begin{minipage}[b]{\linewidth}\raggedright
Range
\end{minipage} & \begin{minipage}[b]{\linewidth}\raggedright
\textbf{Habitat Suitability Value}
\end{minipage} \\
\midrule\noalign{}
\endfirsthead
\toprule\noalign{}
\begin{minipage}[b]{\linewidth}\raggedright
\textbf{Parameter}
\end{minipage} & \begin{minipage}[b]{\linewidth}\raggedright
Range
\end{minipage} & \begin{minipage}[b]{\linewidth}\raggedright
\textbf{Habitat Suitability Value}
\end{minipage} \\
\midrule\noalign{}
\endhead
\bottomrule\noalign{}
\endlastfoot
A. Temperature (°C) & \(0 < t < 8\)

\(8 <= t < 12\)

\(12 <= t < 16\)

\(16 <= t < 22\)

\(22 <= t < 27\)

\(27 <= t < 30\)

\(t > 30\) & 0.0

0.5

1.0

1.0

0.5

0.1

0.0 \\
B. Depth (\emph{meters}) & \(MLT < d < 2\)

\(2 <= d < 10\)

\(10 <= d < 20\)

\(20 <= d < 50\)

\(50 <= d < 100\)

\(d > 100\) & 0.0

0.5

1.0

0.5

0.5

0.5 \\
C. Salinity (\emph{psu}) & \(0 < s < 0.5\)

\(0.5 <= s < 5.0\)

\(5.0 <= s < 15.0\)

\(15.0 <= s < 20.0\)

\(s > 20\) & 1.0

1.0

1.0

0.5

0.0 \\
D. Flow Velocity (\emph{m/s}) & \(0 < v < 0.3\)

\(0.3 <= v < 1.5\)

\(1.5 <= v < 3.5\)

\(3.5 <= v < 4.5\)

\(v > 5\) & 1.0

1.0

0.5

0.3

0.0 \\
E. Substrate & Hard Substrate

Soft Substrate

Present SAV

Absent SAV & 0.5

1.0

1.0

0.5 \\
\end{longtable}

\hypertarget{temperature-7}{%
\subsubsection{Temperature}\label{temperature-7}}

\hypertarget{depth-7}{%
\subsubsection{Depth}\label{depth-7}}

\hypertarget{salinity-7}{%
\subsubsection{Salinity}\label{salinity-7}}

\hypertarget{flow-velocity-7}{%
\subsubsection{Flow Velocity}\label{flow-velocity-7}}

\hypertarget{substrate-7}{%
\subsubsection{Substrate}\label{substrate-7}}

Previous studies have presented conflicting information regarding the substrate preferences of spawning adult blueback herring, likely stemming from the generalization of alewives with blueback herring as river herring.
Adult blueback herring spawn over both soft and hard substrates (Pardue 1983; O'Connell and Angermeier 1997; Brown et al. 2000).

Aside from their demonstrated dependence on soft substrates, spawning adult alewives also exhibit a pronounced inclination toward habitats containing sub-aquatic vegetation (Killgore, Morgan, and Hurley 1988; Laney 1997).
Comprehending and defining these substrate preferences is crucial to effectively manage and conserve the appropriate spawning habitats for alewives.

\hypertarget{larvae}{%
\subsection{Larvae}\label{larvae}}

\begin{longtable}[]{@{}
  >{\raggedright\arraybackslash}p{(\columnwidth - 4\tabcolsep) * \real{0.3494}}
  >{\raggedright\arraybackslash}p{(\columnwidth - 4\tabcolsep) * \real{0.2530}}
  >{\raggedright\arraybackslash}p{(\columnwidth - 4\tabcolsep) * \real{0.3855}}@{}}
\caption{\textbf{Table 3.} Model Parameters and Habitat Suitability Values for Blueback Herring Larvae Stage.}\tabularnewline
\toprule\noalign{}
\begin{minipage}[b]{\linewidth}\raggedright
\textbf{Parameter}
\end{minipage} & \begin{minipage}[b]{\linewidth}\raggedright
Range
\end{minipage} & \begin{minipage}[b]{\linewidth}\raggedright
\textbf{Habitat Suitability Value}
\end{minipage} \\
\midrule\noalign{}
\endfirsthead
\toprule\noalign{}
\begin{minipage}[b]{\linewidth}\raggedright
\textbf{Parameter}
\end{minipage} & \begin{minipage}[b]{\linewidth}\raggedright
Range
\end{minipage} & \begin{minipage}[b]{\linewidth}\raggedright
\textbf{Habitat Suitability Value}
\end{minipage} \\
\midrule\noalign{}
\endhead
\bottomrule\noalign{}
\endlastfoot
A. Temperature (°C) & \(0 < t < 8\)

\(8 <= t < 12\)

\(12 <= t < 16\)

\(16 <= t < 22\)

\(22 <= t < 27\)

\(27 <= t < 30\)

\(t > 30\) & 0.0

0.5

1.0

1.0

0.5

0.1

0.0 \\
B. Depth (\emph{meters}) & \(MLT < d < 2\)

\(2 <= d < 10\)

\(10 <= d < 20\)

\(20 <= d < 50\)

\(50 <= d < 100\)

\(d > 100\) & 0.0

0.5

1.0

0.5

0.5

0.5 \\
C. Salinity (\emph{psu}) & \(0 < s < 0.5\)

\(0.5 <= s < 5.0\)

\(5.0 <= s < 15.0\)

\(15.0 <= s < 20.0\)

\(s > 20\) & 1.0

1.0

1.0

0.5

0.0 \\
D. Flow Velocity (\emph{m/s}) & \(0 < v < 0.3\)

\(0.3 <= v < 1.5\)

\(1.5 <= v < 3.5\)

\(3.5 <= v < 4.5\)

\(v > 5\) & 1.0

1.0

0.5

0.3

0.0 \\
E. Substrate & Hard Substrate

Soft Substrate

Present SAV

Absent SAV & 0.5

1.0

1.0

0.5 \\
\end{longtable}

\hypertarget{temperature-8}{%
\subsubsection{Temperature}\label{temperature-8}}

\hypertarget{depth-8}{%
\subsubsection{Depth}\label{depth-8}}

\hypertarget{salinity-8}{%
\subsubsection{Salinity}\label{salinity-8}}

\hypertarget{flow-velocity-8}{%
\subsubsection{Flow Velocity}\label{flow-velocity-8}}

\hypertarget{substrate-8}{%
\subsubsection{Substrate}\label{substrate-8}}

\hypertarget{results}{%
\chapter{Results}\label{results}}

\hypertarget{model-application}{%
\section{Model Application}\label{model-application}}

example of application to martha's vineyard

\hypertarget{study-area}{%
\subsection{Study Area}\label{study-area}}

We do not know right now the precise
places where the herring spawn within Squibnocket
and its tributary brooks and pond and salinity conditions where spawning occurs. spawn successfully in the slightly brackish water
at the west end of the Pond (or elsewhere within
it), or whether they manage to work their way up
little tributaries. Bourne 1990.

\hypertarget{the-rationale-for-model-application-to-marthas-vineyard}{%
\section{The Rationale for Model Application to Martha's Vineyard}\label{the-rationale-for-model-application-to-marthas-vineyard}}

\begin{itemize}
\tightlist
\item
  Discussing the choice of Martha's Vineyard as the project site.
  (Herring Creek Fishery)
\item
  The relevance of the study area to river herring management.
  Historic Significance, Tribal Significance, and Herring Significance
\item
  Unique features that make Martha's Vineyard an ideal location for model application.
  Historically and maintains a mostly brackish spawning environment, freshawater microhabitats expected in western range of habitat
  (Squibnocket used to be open to the sea until the early 1700s after a storm closed an artifical opening) Natural opening was present before this and replaced by the artificial opening after closing naturally.)
  river herring do not swim up freshwater streams attached to Squibnocket Pond
\end{itemize}

\hypertarget{input-data}{%
\subsection{Input Data}\label{input-data}}

\hypertarget{spawning-adults-2}{%
\subsection{Spawning Adults}\label{spawning-adults-2}}

\begin{itemize}
\tightlist
\item
  Alewives
\item
  Blueback Herring
\end{itemize}

\hypertarget{non-migratory-juveniles-3}{%
\subsection{Non-Migratory Juveniles}\label{non-migratory-juveniles-3}}

\begin{itemize}
\tightlist
\item
  Alewives
\item
  Blueback Herring
\end{itemize}

\hypertarget{eggs-and-larvae-1}{%
\subsection{Eggs and Larvae}\label{eggs-and-larvae-1}}

\begin{itemize}
\tightlist
\item
  Alewives
\item
  Blueback Herring
\end{itemize}

\hypertarget{model-evaluation}{%
\chapter{Model Evaluation}\label{model-evaluation}}

\hypertarget{system-quality}{%
\section{System Quality}\label{system-quality}}

\hypertarget{technical-quality}{%
\section{Technical Quality}\label{technical-quality}}

\hypertarget{usability}{%
\section{Usability}\label{usability}}

\hypertarget{discussion}{%
\chapter{Discussion}\label{discussion}}

\hypertarget{strengths}{%
\section{Strengths}\label{strengths}}

\hypertarget{limitations}{%
\section{Limitations}\label{limitations}}

\begin{itemize}
\item
  regional
\item
  accuracy based on current knowledge
\item
  prediction power
\item
  Compare to similar models (Pardue \& Brown)
\end{itemize}

other factors impacting habitat health that could be added in the future
- nitrogen
- dissolved oxygen
- PH levels
``Blueback eggs and larvae to pulses of acidified water
(pH 5.5 to 5.6) was reported to increase their
mortality rate substantially (Klauda, 1987; see also
Byrne, 1988). The pH of many Cape and Island soils is
roughly in tnis range, and the ability of these soils
to neutralize additional acidic inputs is limited. Bourne 1990)

\hypertarget{implications-for-river-herring}{%
\section{Implications for River Herring}\label{implications-for-river-herring}}

\hypertarget{management-implications}{%
\section{Management Implications}\label{management-implications}}

\begin{itemize}
\item
  Discussion of whether the model is communicated effectively for accurate and easy application in other project locations.
\item
  Are the assumptions and limitations staated clearly
\item
  Is the model clear? (are the breakpoints presented in a way that is clear and understandable)
\item
  Are the results understandable? What do the results mean?

  \begin{itemize}
  \tightlist
  \item
    Previewing the role of the discussion and summary chapters in interpreting findings.
  \end{itemize}
\item
  different life stages have drastically different preferences and therefore independent models should be utilized to accommodate the different resource needs at different life stages for river herring. River herring species should be careful generalizing alewife and blueback herring species together as species exhibit different habitat preferences.

  \begin{itemize}
  \tightlist
  \item
    Emphasizing the contribution of the study to advancing river herring management practices.
    Utlizing multiple models that reflect the conditions for optimal development and spawning of alewives and blueback herring can identify management weaknesses at different life stages. For example, \_\_\_\_ found that although habitat resources were sufficient for juveniles, habita for egg and larvae developement was lacking.
  \end{itemize}
\end{itemize}

\hypertarget{summary}{%
\chapter{Summary}\label{summary}}

\hypertarget{references}{%
\chapter*{References}\label{references}}
\addcontentsline{toc}{chapter}{References}

\hypertarget{appendices}{%
\chapter*{Appendices}\label{appendices}}
\addcontentsline{toc}{chapter}{Appendices}

\hypertarget{refs}{}
\begin{CSLReferences}{1}{0}
\leavevmode\vadjust pre{\hypertarget{ref-able_alewife_2020}{}}%
Able, K. W., T. M. Grothues, M. J. Shaw, S. M. VanMorter, M. C. Sullivan, and D. D. Ambrose. 2020. {``Alewife ({Alosa} Pseudoharengus) Spawning and Nursery Areas in a Sentinel Estuary: Spatial and Temporal Patterns.''} \emph{Environmental Biology of Fishes} 103 (11): 1419--36. \url{https://doi.org/10.1007/s10641-020-01032-0}.

\leavevmode\vadjust pre{\hypertarget{ref-ames_gadids_2013}{}}%
Ames, Edward P., and John Lichter. 2013. {``Gadids and {Alewives}: {Structure} Within Complexity in the {Gulf} of {Maine}.''} \emph{Fisheries Research} 141 (April): 70--78. \url{https://doi.org/10.1016/j.fishres.2012.09.011}.

\leavevmode\vadjust pre{\hypertarget{ref-asmfc_fishery_1985}{}}%
ASMFC. 1985. {``Fishery {Management} {Plan} for {American} {Shad} and {River} {Herrings}.''} Fisheries \{Management\} \{Report\} 6. Washington, D. C. 20036: Atlantic States Marine Fisheries Commission.

\leavevmode\vadjust pre{\hypertarget{ref-asmfc_amendment_2009}{}}%
---------. 2009. {``{AMENDMENT} 2 to the {Interstate} {Fishery} {Management} {Plan} for {SHAD} {AND} {RIVER} {HERRING} ({River} {Herring} {Management}).''} Fisheries \{Management\} \{Report\} 35. Washington, D. C. 20036: Atlantic States Marine Fisheries Commission.

\leavevmode\vadjust pre{\hypertarget{ref-asmfc_river_2017}{}}%
---------. 2017. {``River {Herring} {Stock} {Assessment} {Update} {Volume} {I}: {Coastwide} {Summary}.''} Stock \{Assessment\} \{Report\} 12-02. Washington, D. C.: Atlantic States Marine Fisheries Commission.

\leavevmode\vadjust pre{\hypertarget{ref-bethoney_environmental_2014}{}}%
Bethoney, N. David, Kevin D. E. Stokesbury, and Steven X. Cadrin. 2014. {``Environmental Links to Alosine at-Sea Distribution and Bycatch in the {Northwest} {Atlantic} Midwater Trawl Fishery.''} \emph{ICES Journal of Marine Science} 71 (5): 1246--55. \url{https://doi.org/10.1093/icesjms/fst013}.

\leavevmode\vadjust pre{\hypertarget{ref-bigelow_fishes_1953}{}}%
Bigelow, Henry Bryant, and William Schroeder. 1953. \emph{Fishes of the {Gulf} of {Maine}}. 7135th Series. Fish Bulletin. \url{http://www.gma.org/fogm/}.

\leavevmode\vadjust pre{\hypertarget{ref-boger_development_2002}{}}%
Boger, Rebecca Ann. 2002. {``Development of a Watershed and Stream-Reach Spawning Habitat Model for River Herring {Alosa} Pseudoharengus and {Alosa} Aestivalis.''} PhD thesis, Virginia: College of WIlliam; Mary.

\leavevmode\vadjust pre{\hypertarget{ref-boscarino_influence_2020}{}}%
Boscarino, Brent T., Sonomi Oyagi, Elinor K. Stapylton, Katherine E. McKeon, Noland O. Michels, Susan F. Cushman, and Meghan E. Brown. 2020. {``The Influence of Light, Substrate, and Fish on the Habitat Preferences of the Invasive Bloody Red Shrimp, {Hemimysis} Anomala.''} \emph{Journal of Great Lakes Research} 46 (2): 311--22. \url{https://doi.org/10.1016/j.jglr.2020.01.004}.

\leavevmode\vadjust pre{\hypertarget{ref-bourne_appendix_1990}{}}%
Bourne, Donald. 1990. {``{APPENDIX} 1: {REVIEW} {OF} {THE} {BIOLOGY} {OF} {ALEWIVES} {AND} {BLUEBACK} {HERRING} ({ALOSA} {PSEUOOHARENGUS} {AND} {A}. {AESTIVALIS}) {AND} {THE} {FISHERIES}, {WITH} {REFERENCE} {TO} {SQUIBNOCKET} {POND}, {MARTHA}'{S} {VINEYARD}.''} Woods Hole, Massachusetts.

\leavevmode\vadjust pre{\hypertarget{ref-brady_part_2005}{}}%
Brady, P. D., Kenneth E. Reback, Katherine D. McLaughlin, and Cheryl Milliken. 2005. {``Part 4. {Boston} {Harbor}, {North} {Shore}, and {Merrimack} {River}.''} Technical \{Report\}. Pocasset, MA: Massachusetts Division of Marine Fisheries.

\leavevmode\vadjust pre{\hypertarget{ref-brown_habitat_2000}{}}%
Brown, Stephen K., Kenneth R. Buja, Steven H. Jury, Mark E. Monaco, and Arnold Banner. 2000. {``Habitat {Suitability} {Index} {Models} for {Eight} {Fish} and {Invertebrate} {Species} in {Casco} and {Sheepscot} {Bays}, {Maine}.''} \emph{North American Journal of Fisheries Management} 20 (2): 408--35. \url{https://doi.org/10.1577/1548-8675(2000)020\%3C0408:HSIMFE\%3E2.3.CO;2}.

\leavevmode\vadjust pre{\hypertarget{ref-christensen_branchial_2012}{}}%
Christensen, A. K., J. Hiroi, E. T. Schultz, and S. D. McCormick. 2012. {``Branchial Ionocyte Organization and Ion-Transport Protein Expression in Juvenile Alewives Acclimated to Freshwater or Seawater.''} \emph{Journal of Experimental Biology} 215 (4): 642--52. \url{https://doi.org/10.1242/jeb.063057}.

\leavevmode\vadjust pre{\hypertarget{ref-cianci_larval_1969}{}}%
Cianci, J. M. 1969. {``Larval {Development} of the Alewife and the Glut Herring.''} Master's thesis, University of Connecticut.

\leavevmode\vadjust pre{\hypertarget{ref-collette_fishes_2002}{}}%
Collette, Bruce, and Grace Klein-MacPhee. 2002. {``Fishes of the {Gulf} of {Maine} for the 21st {Century}: {A} {Look} at the {New} {Bigelow} and {Schroeder}.''} \emph{BioScience} 53 (8): 772. \url{https://doi.org/10.1641/0006-3568(2003)053\%5B0772:FOTGOM\%5D2.0.CO;2}.

\leavevmode\vadjust pre{\hypertarget{ref-creaser_distribution_1994}{}}%
Creaser, E. P., and H. C. Perkins. 1994. {``The Distribution, Food, and Age of Juvenile Bluefish, {Pomatomus} Saltatrix, in {Maine}''} 99: 494--508.

\leavevmode\vadjust pre{\hypertarget{ref-dimaggio_effects_2016}{}}%
DiMaggio, Matthew A., Timothy S. Breton, Linas W. Kenter, Calvin G. Diessner, Aurora I. Burgess, and David L. Berlinsky. 2016. {``The Effects of Elevated Salinity on River Herring Embryo and Larval Survival.''} \emph{Environmental Biology of Fishes} 99 (5): 451--61. \url{https://doi.org/10.1007/s10641-016-0488-7}.

\leavevmode\vadjust pre{\hypertarget{ref-dimaggio_spawning_2015}{}}%
DiMaggio, Matthew A., Harvey J. Pine, Linas W. Kenter, and David L. Berlinsky. 2015. {``Spawning, {Larviculture}, and {Salinity} {Tolerance} of {Alewives} and {Blueback} {Herring} in {Captivity}.''} \emph{North American Journal of Aquaculture} 77 (3): 302--11. \url{https://doi.org/10.1080/15222055.2015.1009590}.

\leavevmode\vadjust pre{\hypertarget{ref-dovel_recent_1971}{}}%
Dovel, William L., and James R. Edmunds. 1971. {``Recent {Changes} in {Striped} {Bass} ({Morone} Saxatilis) {Spawning} {Sites} and {Commercial} {Fishing} {Areas} in {Upper} {Chesapeake} {Bay}; {Possible} {Influencing} {Factors}.''} \emph{Chesapeake Science} 12 (1): 33. \url{https://doi.org/10.2307/1350500}.

\leavevmode\vadjust pre{\hypertarget{ref-dunstall_distribution_1984}{}}%
Dunstall, Thomas G. 1984. {``Distribution of {Rainbow} {Smelt} and {Alewife} {Larvae} {Along} the {North} {Shore} of {Lake} {Ontario}.''} \emph{Journal of Great Lakes Research} 10 (3): 273--79. \url{https://doi.org/10.1016/S0380-1330(84)71840-5}.

\leavevmode\vadjust pre{\hypertarget{ref-durbin_effects_1979}{}}%
Durbin, Ann Gall, Scott W. Nixon, and Candace A. Oviatt. 1979. {``Effects of the {Spawning} {Migration} of the {Alewife}, {Alosa} {Pseudoharengus}, on {Freshwater} {Ecosystems}.''} \emph{Ecology} 60 (1): 8--17. \url{https://doi.org/10.2307/1936461}.

\leavevmode\vadjust pre{\hypertarget{ref-esdall_feeding_1964}{}}%
Esdall, T. A. 1964. {``Feeding by Three Species of Fishes on the Eggs of Spawning Alewives.''}

\leavevmode\vadjust pre{\hypertarget{ref-esdall_effect_1970}{}}%
---------. 1970. {``The Effect of Temperature on the Rate of Development and Survival of Alewife Eggs and Larvae''} 99.

\leavevmode\vadjust pre{\hypertarget{ref-fay_alewifeblueback_1983}{}}%
Fay, Clemon, Richard Neves, and Garland Pardue. 1983. {``Alewife/{Blueback} {Herring}.''} Biological \{Report\} FWS/OBS-82/11.9. Blacksburg, VA: U.S. Fish; WIldlife Service, Division of Biological Sciences, U.S. Army Corps of Engineers. \url{https://apps.dtic.mil/sti/tr/pdf/ADA180383.pdf}.

\leavevmode\vadjust pre{\hypertarget{ref-ford_herring_1929}{}}%
Ford, E. 1929. {``Herring {Investigations} at {Plymouth}. {VII}. {On} the {Artificial} {Fertilisation} and {Hatching} of {Herring} {Eggs} Under Known {Conditions} of {Salinity}, with Some {Observations} on the {Specific} {Gravity} of the {Larvæ}.''} \emph{Journal of the Marine Biological Association of the United Kingdom} 16 (1): 43--48. \url{https://doi.org/10.1017/S0025315400029684}.

\leavevmode\vadjust pre{\hypertarget{ref-frank_role_2011}{}}%
Frank, H. J., M. E. Mather, J. M. Smith, R. M. Muth, and J. T. Finn. 2011. {``Role of Origin and Release Location in Pre-Spawning Distribution and Movements of Anadromous Alewife: {PRE}-{SPAWNING} {ALEWIFE} {DISTRIBUTION} {AND} {MOVEMENT}.''} \emph{Fisheries Management and Ecology} 18 (1): 12--24. \url{https://doi.org/10.1111/j.1365-2400.2010.00759.x}.

\leavevmode\vadjust pre{\hypertarget{ref-greene_atlantic_2009}{}}%
Greene, K. E., J. L. Zimmerman, R. W. Laney, and Jessie Thomas-Blate. 2009. {``Atlantic {Coast} {Diadromous} {Fish} {Habitat}: {A} {Review} of {Utilization}, {Threats}, {Recommendations} for {Conservation}, and {Research} {Needs}.''} Atlantic States Marine Fisheries Commission. \url{chrome-extension://efaidnbmnnnibpcajpcglclefindmkaj/https://www3.epa.gov/region1/npdes/merrimackstation/pdfs/ar/AR-56.pdf}.

\leavevmode\vadjust pre{\hypertarget{ref-haegele_distribution_1985}{}}%
Haegele, C. W., and J. F. Schweigert. 1985. {``Distribution and {Characteristics} of {Herring} {Spawning} {Grounds} and {Description} of {Spawning} {Behavior}.''} \emph{Canadian Journal of Fisheries and Aquatic Sciences} 42 (S1): s39--55. \url{https://doi.org/10.1139/f85-261}.

\leavevmode\vadjust pre{\hypertarget{ref-haro_swimming_2004}{}}%
Haro, Alex, Theodore Castro-Santos, John Noreika, and Mufeed Odeh. 2004. {``Swimming Performance of Upstream Migrant Fishes in Open-Channel Flow: A New Approach to Predicting Passage Through Velocity Barriers.''} \emph{Canadian Journal of Fisheries and Aquatic Sciences} 61 (9): 1590--1601. \url{https://doi.org/10.1139/f04-093}.

\leavevmode\vadjust pre{\hypertarget{ref-henderson_effects_1985}{}}%
Henderson, Bryan A., and Edward H. Brown Jr. 1985. {``Effects of {Abundance} and {Water} {Temperature} on {Recruitment} and {Growth} of {Alewife} ( \emph{{Alosa} Pseudoharengus} ) Near {South} {Bay}, {Lake} {Huron}, 1954--82.''} \emph{Canadian Journal of Fisheries and Aquatic Sciences} 42 (10): 1608--13. \url{https://doi.org/10.1139/f85-201}.

\leavevmode\vadjust pre{\hypertarget{ref-hook_annual_2008}{}}%
Höök, Tomas O., Edward S. Rutherford, Thomas E. Croley, Doran M. Mason, and Charles P. Madenjian. 2008. {``Annual Variation in Habitat-Specific Recruitment Success: Implications from an Individual-Based Model of {Lake} {Michigan} Alewife ({Alosa} Pseudoharengus).''} \emph{Canadian Journal of Fisheries and Aquatic Sciences} 65 (7): 1402--12. \url{https://doi.org/10.1139/F08-066}.

\leavevmode\vadjust pre{\hypertarget{ref-ingel_habitat_2013}{}}%
Ingel, Claire. 2013. {``Habitat {Use}, {Growth}, and {Feeding} of {Larval} {Alewife} in a {Shallow} {River} {Margin} of the {Upper} {Hudson} {River}.''} Master's thesis, Cornell University. \url{https://ecommons.cornell.edu/bitstream/handle/1813/33829/ces279.pdf?sequence=1\&isAllowed=y}.

\leavevmode\vadjust pre{\hypertarget{ref-janssen_will_1978}{}}%
Janssen, John. 1978. {``Will Alewives ({Alosa} Pseudoharengus) Feed in the Dark?''} \emph{Environmental Biology of Fishes} 3 (2): 239--40. \url{https://doi.org/10.1007/BF00691949}.

\leavevmode\vadjust pre{\hypertarget{ref-janssen_feeding_1980}{}}%
Janssen, John, and Stephen B. Brandt. 1980. {``Feeding {Ecology} and {Vertical} {Migration} of {Adult} {Alewives} ( \emph{{Alosa} Pseudoharengus} ) in {Lake} {Michigan}.''} \emph{Canadian Journal of Fisheries and Aquatic Sciences} 37 (2): 177--84. \url{https://doi.org/10.1139/f80-023}.

\leavevmode\vadjust pre{\hypertarget{ref-janssen_preference_2004}{}}%
Janssen, John, and Michelle A. Luebke. 2004. {``Preference for {Rocky} {Habitat} by {Age}-0 {Yellow} {Perch} and {Alewives}.''} \emph{Journal of Great Lakes Research} 30 (1): 93--99. \url{https://doi.org/10.1016/S0380-1330(04)70332-9}.

\leavevmode\vadjust pre{\hypertarget{ref-kellogg_temperature_1982}{}}%
Kellogg, Robert L. 1982. {``Temperature {Requirements} for the {Survival} and {Early} {Development} of the {Anadromous} {Alewife}.''} \emph{The Progressive Fish-Culturist} 44 (2): 63--73. \url{https://doi.org/10.1577/1548-8659(1982)44\%5B63:TRFTSA\%5D2.0.CO;2}.

\leavevmode\vadjust pre{\hypertarget{ref-killgore_distribution_1988}{}}%
Killgore, K. Jack, Raymond P. Morgan, and Linda M. Hurley. 1988. {``Distribution and {Abundance} of {Fishes} in {Aquatic} {Vegetation}.''} Miscellaneous \{Paper\} \{A\}-87-2 88-11-15-001. Vicksburg, Mississippi: Engineer Research; Development Center (U.S.).

\leavevmode\vadjust pre{\hypertarget{ref-kissil_spawning_1974}{}}%
Kissil, George William. 1974. {``Spawning of the {Anadromous} {Alewife}, {Alosa} Pseudoharengus, in {Bride} {Lake}, {Connecticut}.''} \emph{Transactions of the American Fisheries Society} 103 (2): 312--17. \url{https://doi.org/10.1577/1548-8659(1974)103\%3C312:SOTAAA\%3E2.0.CO;2}.

\leavevmode\vadjust pre{\hypertarget{ref-klauda_alewife_1991}{}}%
Klauda, R. J., L. W. Hall Fischer, and J. A. Sullivan. 1991. {``Alewife and {Blueback} {Herring}, {Alosa} Pseudoharengus and {Alosa} Aestivalis.''} Solomons, Maryland.

\leavevmode\vadjust pre{\hypertarget{ref-klumb_importance_2003}{}}%
Klumb, Robert A., Lars G. Rudstam, Edward L. Mills, Clifford P. Schneider, and Paul M. Sawyko. 2003. {``Importance of {Lake} {Ontario} {Embayments} and {Nearshore} {Habitats} as {Nurseries} for {Larval} {Fishes} with {Emphasis} on {Alewife} ({Alosa} Pseudoharengus).''} \emph{Journal of Great Lakes Research} 29 (1): 181--98. \url{https://doi.org/10.1016/S0380-1330(03)70426-2}.

\leavevmode\vadjust pre{\hypertarget{ref-kocovsky_linking_2008}{}}%
Kocovsky, Patrick M., Robert M. Ross, David S. Dropkin, and John M. Campbell. 2008. {``Linking {Landscapes} and {Habitat} {Suitability} {Scores} for {Diadromous} {Fish} {Restoration} in the {Susquehanna} {River} {Basin}.''} \emph{North American Journal of Fisheries Management} 28 (3): 906--18. \url{https://doi.org/10.1577/M06-120.1}.

\leavevmode\vadjust pre{\hypertarget{ref-kornis_linking_2011}{}}%
Kornis, Matthew S., and John Janssen. 2011. {``Linking Emergent Midges to Alewife ({Alosa} Pseudoharengus) Preference for Rocky Habitat in {Lake} {Michigan} Littoral Zones.''} \emph{Journal of Great Lakes Research} 37 (3): 561--66. \url{https://doi.org/10.1016/j.jglr.2011.05.009}.

\leavevmode\vadjust pre{\hypertarget{ref-kosa_processes_2001}{}}%
Kosa, Jarrad T., and Martha E. Mather. 2001. {``Processes {Contributing} to {Variability} in {Regional} {Patterns} of {Juvenile} {River} {Herring} {Abundance} Across {Small} {Coastal} {Systems}.''} \emph{Transactions of the American Fisheries Society} 130 (4): 600--619. \url{https://doi.org/10.1577/1548-8659(2001)130\%3C0600:PCTVIR\%3E2.0.CO;2}.

\leavevmode\vadjust pre{\hypertarget{ref-laney_relationship_1997}{}}%
Laney, R. Wilson. 1997. {``The {Relationship} of {Submerged} {Aquatic} {Vegetation} ({SAV}) {Ecological} {Value} to {Species} {Managed} by the {Atlantic} {States} {Marine} {Fisheries} {Commission} ({ASMFC}): {Summary} for the {ASMFC} {SAV} {Subcommittee}.''} \{ASMFC\} \{Habitat\} \{Management\} \{Series\} \textbackslash\#1. South Atlantic Fisheries Resources Coordination Office, Raleigh, NC: U.S. Fish; Wildlife Service, Southeast Region, Southeast Region.

\leavevmode\vadjust pre{\hypertarget{ref-legett_daily_2021}{}}%
Legett, Henry D., Adrian Jordaan, Allison H. Roy, John J. Sheppard, Marcelo Somos‐Valenzuela, and Michelle D. Staudinger. 2021. {``Daily {Patterns} of {River} {Herring} ( \emph{Alosa} Spp.) {Spawning} {Migrations}: {Environmental} {Drivers} and {Variation} Among {Coastal} {Streams} in {Massachusetts}.''} \emph{Transactions of the American Fisheries Society} 150 (4): 501--13. \url{https://doi.org/10.1002/tafs.10301}.

\leavevmode\vadjust pre{\hypertarget{ref-leim_life_1924}{}}%
Leim, Alexander. 1924. \emph{The {Life} {History} of the {Shad} ({Alosa} {Sapidissima} ({Wilson})) with {Special} {Reference} to the {Factors} {Limiting} {Its} {Abundance}}. Univeristy of Toronto Press.

\leavevmode\vadjust pre{\hypertarget{ref-loesch_overview_1987}{}}%
Loesch, J. G. 1987. {``Overview of {Life} {History} {Aspects} of {Anadromous} {Alewife} and {Blueback} {Herring} in {Freshwater} {Habitats}.''} \url{https://scholarworks.umass.edu/fishpassage_journal_articles/10/}.

\leavevmode\vadjust pre{\hypertarget{ref-lynch_projected_2015}{}}%
Lynch, Patrick D., Janet A. Nye, Jonathan A. Hare, Charles A. Stock, Michael A. Alexander, James D. Scott, Kiersten L. Curti, and Katherine Drew. 2015. {``Projected Ocean Warming Creates a Conservation Challenge for River Herring Populations.''} \emph{ICES Journal of Marine Science} 72 (2): 374--87. \url{https://doi.org/10.1093/icesjms/fsu134}.

\leavevmode\vadjust pre{\hypertarget{ref-mather_assessing_2012}{}}%
Mather, Martha E., Holly J. Frank, Joseph M. Smith, Roxann D. Cormier, Robert M. Muth, and John T. Finn. 2012. {``Assessing {Freshwater} {Habitat} of {Adult} {Anadromous} {Alewives} {Using} {Multiple} {Approaches}.''} \emph{Marine and Coastal Fisheries} 4 (1): 188--200. \url{https://doi.org/10.1080/19425120.2012.675980}.

\leavevmode\vadjust pre{\hypertarget{ref-mccartin_new_2019}{}}%
McCartin, Kellie, Adrian Jordaan, Matthew Sclafani, Robert Cerrato, and Michael G. Frisk. 2019. {``A {New} {Paradigm} in {Alewife} {Migration}: {Oscillations} Between {Spawning} {Grounds} and {Estuarine} {Habitats}.''} \emph{Transactions of the American Fisheries Society} 148 (3): 605--19. \url{https://doi.org/10.1002/tafs.10155}.

\leavevmode\vadjust pre{\hypertarget{ref-mullen_species_1986}{}}%
Mullen, D. M., C. W. Fay, and J. R. Moring. 1986. {``Species Profiles: Life Histories and Environmental Requirements of Coastal Fishes and Invertebrates ({North} {Atlantic})--Alewife/Blueback Herring.''} U.\{S\}. \{Fish\} and \{Wildlife\} \{Service\} \{Biological\} \{Report\} 82(11.56). USACE.

\leavevmode\vadjust pre{\hypertarget{ref-munroe_overview_2000}{}}%
Munroe, Thomas. 2000. {``An Overview of the Biology, Ecology, and Fisheries of the Clupeoid Fishes Occurring in the {Gulf} of {Maine}.''} Reference \{Document\} 00-02. Woods Hole, Massachusetts: National Matine Fisheries Service, Northeast Fisheries Science Center. \url{https://repository.library.noaa.gov/view/noaa/5081}.

\leavevmode\vadjust pre{\hypertarget{ref-murdy_fishes_1997}{}}%
Murdy, Edward O., Ray S. Birdsong, and John A. Musick. 1997. \emph{Fishes of the {Chesapeake} {Bay}}. Washington; London: Smithsonian Institution Press.

\leavevmode\vadjust pre{\hypertarget{ref-nmfs_national_marine_fisheries_service_species_2009}{}}%
National Marine Fisheries Service, (NMFS). 2009. {``Species of Concern: River Herring.''} \href{https://nmfs.noaa.gov/pr/species/concern/}{nmfs.noaa.gov/pr/species/concern/}.

\leavevmode\vadjust pre{\hypertarget{ref-nmfs_national_marine_fisheries_service_endangered_2013}{}}%
---------. 2013. {``Endangered and Threatened Wildlife and Plants; Endangered Species Act Listing Determination for Alewife and Blueback Herring.''}

\leavevmode\vadjust pre{\hypertarget{ref-nmfs_national_marine_fisheries_service_not_2019}{}}%
---------. 2019. {``Not {Warented} {Listing} {Determination}.''}

\leavevmode\vadjust pre{\hypertarget{ref-oconnell_spawning_1997}{}}%
O'Connell, Ann M. (Uzee), and Paul L. Angermeier. 1997. {``Spawning {Location} and {Distribution} of {Early} {Life} {Stages} of {Alewife} and {Blueback} {Herring} in a {Virginia} {Stream}.''} \emph{Estuaries} 20 (4): 779. \url{https://doi.org/10.2307/1352251}.

\leavevmode\vadjust pre{\hypertarget{ref-oconnell_habitat_1999}{}}%
---------. 1999. {``Habitat {Relationships} for {Alewife} and {Blueback} {Herring} {Spawning} in a {Virginia} {Stream}.''} \emph{Journal of Freshwater Ecology} 14 (3): 357--70. \url{https://doi.org/10.1080/02705060.1999.9663691}.

\leavevmode\vadjust pre{\hypertarget{ref-olney_nearshore_1988}{}}%
Olney, Je, and Gw Boehlert. 1988. {``Nearshore Ichthyoplankton Associated with Seagrass Beds in the Lower {Chesapeake} {Bay}.''} \emph{Marine Ecology Progress Series} 45: 33--43. \url{https://doi.org/10.3354/meps045033}.

\leavevmode\vadjust pre{\hypertarget{ref-otto_lethal_1976}{}}%
Otto, Robert G., Max A. Kitchel, and John O'Hara Rice. 1976. {``Lethal and {Preferred} {Temperatures} of the {Alewife} ({Alosa} Pseudoharengus) in {Lake} {Michigan}.''} \emph{Transactions of the American Fisheries Society} 105 (1): 96--106. \url{https://doi.org/10.1577/1548-8659(1976)105\%3C96:LAPTOT\%3E2.0.CO;2}.

\leavevmode\vadjust pre{\hypertarget{ref-overton_spatial_2012}{}}%
Overton, Anthony S., Nicholas A. Jones, and Roger Rulifson. 2012. {``Spatial and {Temporal} {Variability} in {Instantaneous} {Growth}, {Mortality}, and {Recruitment} of {Larval} {River} {Herring} in {Tar}--{Pamlico} {River}, {North} {Carolina}.''} \emph{Marine and Coastal Fisheries} 4 (1): 218--27. \url{https://doi.org/10.1080/19425120.2012.675976}.

\leavevmode\vadjust pre{\hypertarget{ref-pardue_habitat_1983}{}}%
Pardue, Garland. 1983. {``Habitat {Suitability} {Index} {Models}: {Alewife} and {Blueback} {Herring}.''} \{FWS\}/\{OBS\} 82/10.58. Department of Interior, Fish; Wildlife Service. \url{https://books.google.com/books?hl=en\&lr=\&id=WpTBLRItqHYC\&oi=fnd\&pg=PR6\&dq=Habitat+Suitability+for+Alewives\&ots=Rh70Hi2dbQ\&sig=mWMhRZ5FcP--mJX1NxJuFuZkhoM\#v=onepage\&q=Habitat\%20Suitability\%20for\%20Alewives\&f=false}.

\leavevmode\vadjust pre{\hypertarget{ref-portner_physiology_2008}{}}%
Pörtner, Hans O., and Anthony P. Farrell. 2008. {``Physiology and {Climate} {Change}.''} \emph{Science} 322 (5902): 690--92. \url{https://doi.org/10.1126/science.1163156}.

\leavevmode\vadjust pre{\hypertarget{ref-prendergast_physical_2019}{}}%
Prendergast, Sara. 2019. {``Physical and Biological Influences on Daily Growth Rate of Larval Alewife ({Alosa} Pseudoharengus) in {Lake} {Michigan}.''} Master of \{Science\}, University of Michigan. \url{https://deepblue.lib.umich.edu/handle/2027.42/150628}.

\leavevmode\vadjust pre{\hypertarget{ref-reback_survey_2004}{}}%
Reback, Kenneth E., Phillips Brady, Katherine D. McLaughlin, and Cheryl Milliken. 2004. {``A {Survey} of {Anadromous} {Fish} {Passage} in {Coastal} {Massachusetts}. {Part} 1, {Southern} {Massachusetts}.''} Massachusetts: Massachusetts Division of Marine Fisheries.

\leavevmode\vadjust pre{\hypertarget{ref-richkus_response_1975}{}}%
Richkus, William A. 1975. {``The {Response} of {Juvenile} {Alewives} to {Water} {Currents} in an {Experimental} {Chamber}.''} \emph{Transactions of the American Fisheries Society} 104 (3): 494--98. \url{https://doi.org/10.1577/1548-8659(1975)104\%3C494:TROJAT\%3E2.0.CO;2}.

\leavevmode\vadjust pre{\hypertarget{ref-schmidt_communities_1988}{}}%
Schmidt, R. E., and E. Kiviat. 1988. {``Communities of {Larval} and {Juvenile} {Fish} {Associated} with {Water}-{Chestnut} {Watermilfoil} and {Water}-{Celery} in the {Tivoli} {Bays} of the {Hudson} {River}.''} Technical \{Report\}. Hudsonia Ltd., Annandale, NY.; Hudson River Foundation for Science; Environmental Research, Inc.

\leavevmode\vadjust pre{\hypertarget{ref-smith_overlapping_2015}{}}%
Smith, M. Chad, and Roger A. Rulifson. 2015. {``Overlapping {Habitat} {Use} of {Multiple} {Anadromous} {Fish} {Species} in a {Restricted} {Coastal} {Watershed}.''} \emph{Transactions of the American Fisheries Society} 144 (6): 1173--83. \url{https://doi.org/10.1080/00028487.2015.1074617}.

\leavevmode\vadjust pre{\hypertarget{ref-stevens_evidence_2021}{}}%
Stevens, Justin R., Rory Saunders, and William Duffy. 2021. {``Evidence of {Life} {Cycle} {Diversity} of {River} {Herring} in the {Penobscot} {River} {Estuary}, {Maine}.''} \emph{Marine and Coastal Fisheries} 13 (3): 292--305. \url{https://doi.org/10.1002/mcf2.10157}.

\leavevmode\vadjust pre{\hypertarget{ref-tommasi_effect_2015}{}}%
Tommasi, Désirée, Janet Nye, Charles Stock, Jonathan A. Hare, Michael Alexander, and Katie Drew. 2015. {``Effect of Environmental Conditions on Juvenile Recruitment of Alewife ( \emph{{Alosa} Pseudoharengus} ) and Blueback Herring ( \emph{{Alosa} Aestivalis} ) in Fresh Water: A Coastwide Perspective.''} Edited by Keith Tierney. \emph{Canadian Journal of Fisheries and Aquatic Sciences} 72 (7): 1037--47. \url{https://doi.org/10.1139/cjfas-2014-0259}.

\leavevmode\vadjust pre{\hypertarget{ref-turner_juvenile_2016}{}}%
Turner, Sara M., and Karin E. Limburg. 2016. {``Juvenile River Herring Habitat Use and Marine Emigration Trends: Comparing Populations.''} \emph{Oecologia} 180 (1): 77--89. \url{https://doi.org/10.1007/s00442-015-3443-y}.

\leavevmode\vadjust pre{\hypertarget{ref-tyus_movements_1974}{}}%
Tyus, Harold M. 1974. {``Movements and {Spawning} of {Anadromous} {Alewives}, {Alosa} Pseudoharengus ({Wilson}) at {Lake} {Mattamuskeet}, {North} {Carolina}.''} \emph{Transactions of the American Fisheries Society} 103 (2): 392--96. \url{https://doi.org/10.1577/1548-8659(1974)103\%3C392:MASOAA\%3E2.0.CO;2}.

\leavevmode\vadjust pre{\hypertarget{ref-waldman_north_2022}{}}%
Waldman, John R., and Thomas P. Quinn. 2022. {``North {American} Diadromous Fishes: {Drivers} of Decline and Potential for Recovery in the {Anthropocene}.''} \emph{Science Advances} 8 (4): eabl5486. \url{https://doi.org/10.1126/sciadv.abl5486}.

\leavevmode\vadjust pre{\hypertarget{ref-walsh_early_2005}{}}%
Walsh, Harvey J., Lawrence R. Settle, and David S. Peters. 2005. {``Early {Life} {History} of {Blueback} {Herring} and {Alewife} in the {Lower} {Roanoke} {River}, {North} {Carolina}.''} \emph{Transactions of the American Fisheries Society} 134 (4): 910--26. \url{https://doi.org/10.1577/T04-060.1}.

\end{CSLReferences}

\end{document}
